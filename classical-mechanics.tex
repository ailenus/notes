\documentclass[12pt]{article}

\makeatletter
\def\@maketitle {
  \newpage
  \null
  \vskip 2em
  \begin{center}
    \let \footnote \thanks
    {
      \Large\bfseries \MakeUppercase{\@title} \par
    }
    \vskip 1.5em
    {
      \large \lineskip .5em \begin{tabular}[t]{c}
        \MakeUppercase{\@author}
      \end{tabular} \par
    }
    \vskip 1.5em
    {
      \large\scshape \@date
    }
  \end{center}
  \par
  \vskip 1.5em
}
\makeatother

%%%%%%%%%%%%%%%%%%%%%%%%%%%%%%%%%%%%%%%%%%%%%%%%%%%%%%%%%%%%%%%%%%%%%%%%%%%%%%%
%% Packages
%%%%%%%%%%%%%%%%%%%%%%%%%%%%%%%%%%%%%%%%%%%%%%%%%%%%%%%%%%%%%%%%%%%%%%%%%%%%%%%

%% basic
\usepackage[a4paper,margin=1in]{geometry}
\usepackage{fontspec}
\usepackage{polyglossia}

%% formatting
\usepackage{setspace}
\usepackage{enumitem}
\usepackage{tocloft}
\usepackage{titlesec}
\usepackage{fancyhdr}

%% maths
\usepackage{amsmath}
\usepackage{amsthm}
\usepackage{mathtools}
\usepackage[math-style=ISO,bold-style=ISO]{unicode-math}

%% utilities
\usepackage{array}
\usepackage{booktabs}
\usepackage[colorlinks=true]{hyperref}

%%%%%%%%%%%%%%%%%%%%%%%%%%%%%%%%%%%%%%%%%%%%%%%%%%%%%%%%%%%%%%%%%%%%%%%%%%%%%%%
%% Settings
%%%%%%%%%%%%%%%%%%%%%%%%%%%%%%%%%%%%%%%%%%%%%%%%%%%%%%%%%%%%%%%%%%%%%%%%%%%%%%%

%% basic
\setmainlanguage[variant=british]{english}
\SetLanguageKeys{english}{indentfirst=true}
\setmainfont{STIX Two Text}
\setmathfont{STIX Two Math}

%% formatting
\doublespacing
\setlist{noitemsep}
\setlist[1]{labelindent=\parindent}
\setlist[1]{listparindent=\parindent}
\setlist[enumerate,1]{label=(\alph*)}
\setlist[enumerate,2]{label=(\roman*)}
\setlength{\headheight}{15.2pt}
\setlength{\cftsecindent}{0em}
\setlength{\cftsubsecindent}{0em}
\setlength{\cftsubsubsecindent}{0em}
\renewcommand*{\cfttoctitlefont}{\large\scshape}
\renewcommand*{\cftsecfont}{\normalfont}
\renewcommand*{\cftsecpagefont}{\normalfont}
\renewcommand*{\cftsecleader}{\cftdotfill{\cftdotsep}}
\titleformat{\section}{
  \normalfont\Large\scshape\filcenter
}{}{1em}{}
\titleformat{\subsection}{
  \normalfont\large\scshape\raggedright
}{\thesubsection}{1em}{}
\titleformat{\subsubsection}{
  \normalfont\normalsize\scshape\raggedright
}{\thesubsubsection}{1em}{}
\setcounter{footnote}{1}
\DeclareEmphSequence{\bfseries\itshape}

%% maths
\makeatletter
\newtheoremstyle{dfn}
  {\topsep}
  {\topsep}
  {\upshape\addtolength{\@totalleftmargin}{3em}
  \addtolength{\linewidth}{-6em}
  \parshape 1 3em \linewidth}
  {0pt}
  {\scshape}
  {.}
  {5pt plus 1pt minus 1pt}
  {}
\newtheoremstyle{thm}
  {\topsep}
  {\topsep}
  {\slshape\addtolength{\@totalleftmargin}{3em}
  \addtolength{\linewidth}{-6em}
  \parshape 1 3em \linewidth}
  {0pt}
  {\bfseries}
  {.}
  {5pt plus 1pt minus 1pt}
  {}
\newtheoremstyle{xmp}
  {\topsep}
  {\topsep}
  {\upshape\addtolength{\@totalleftmargin}{3em}
  \addtolength{\linewidth}{-6em}
  \parshape 1 3em \linewidth}
  {0pt}
  {\bfseries}
  {.}
  {5pt plus 1pt minus 1pt}
  {}
\newtheoremstyle{xrc}
  {\topsep}
  {\topsep}
  {\upshape\addtolength{\@totalleftmargin}{3em}
  \addtolength{\linewidth}{-6em}
  \parshape 1 3em \linewidth}
  {0pt}
  {\bfseries\scshape}
  {.}
  {5pt plus 1pt minus 1pt}
  {}
\newtheoremstyle{sol}
  {\topsep}
  {\topsep}
  {\upshape}
  {0pt}
  {\itshape}
  {.}
  {5pt plus 1pt minus 1pt}
  {}
\makeatother
\theoremstyle{dfn}
\newtheorem{dfn}{Definition}
\theoremstyle{thm}
\newtheorem{thm}{Theorem}
\newtheorem{lem}{Lemma}
\newtheorem{cor}{Corollary}
\renewcommand\qedsymbol{\(\mdlgwhtsquare\)}
\theoremstyle{xmp}
\newtheorem{xmp}{Example}
\theoremstyle{xrc}
\newtheorem{xrc}{Exercise}
\theoremstyle{sol}
\newtheorem*{XxmpX}{Solution}
\newenvironment{solution}{
  \renewcommand{\qedsymbol}{\(\mdlgwhtlozenge\)}
  \pushQED{\qed}\begin{XxmpX}
}{\popQED\end{XxmpX}}
\def\equationautorefname~#1\null{(#1)\null}
\newcommand{\dfnautorefname}{definition}
\newcommand{\thmautorefname}{theorem}
\newcommand{\lemautorefname}{lemma}
\newcommand{\corautorefname}{corollary}

%% aliases
\newcommand*{\Bdc}[2][\today]{
  \title{#2}
  \author{Yannan Mao}
  \date{#1}
  \lhead{\scshape #2}
  \rhead{\scshape Yannan Mao}
  \begin{document}
  \maketitle
  \tableofcontents
  \thispagestyle{empty}
  \clearpage
  \pagestyle{fancy}
  \setcounter{page}{1}
}
\newcommand*{\Edc}{\end{document}}
\newcommand*{\Bdf}{\begin{dfn}}
\newcommand*{\Edf}{\end{dfn}}
\newcommand*{\Bth}{\begin{thm}}
\newcommand*{\Eth}{\end{thm}}
\newcommand*{\Blm}{\begin{lem}}
\newcommand*{\Elm}{\end{lem}}
\newcommand*{\Bcr}{\begin{cor}}
\newcommand*{\Ecr}{\end{cor}}
\newcommand*{\Bxm}{\begin{xmp}}
\newcommand*{\Exm}{\end{xmp}}
\newcommand*{\Bxr}{\begin{xrc}}
\newcommand*{\Exr}{\end{xrc}}
\newcommand*{\Bpr}{\begin{proof}}
\newcommand*{\Epr}{\end{proof}}
\newcommand*{\Bsl}{\begin{solution}}
\newcommand*{\Esl}{\end{solution}}

\DeclareMathOperator{\dom}{dom}           % domain
\DeclareMathOperator{\im}{im}             % image
\DeclareMathOperator*{\argmax}{arg\,max}  % arg max
\DeclareMathOperator*{\argmin}{arg\,min}  % arg min

\DeclarePairedDelimiter{\abs}{\lvert}{\rvert}               % absolute value, cardinality
\DeclarePairedDelimiter{\norm}{\lVert}{\rVert}              % norm
\DeclarePairedDelimiterX{\inn}[2]{\langle}{\rangle}{#1,#2}  % inner product
\DeclarePairedDelimiter{\set}{\{}{\}}                       % set

\newcommand*{\ee}{\ensuremath{\mathrm{e}}}              % e
\newcommand*{\ii}{\ensuremath{\mathrm{i}}}              % i
\newcommand*{\diff}{\ensuremath{\mathop{}\!\mathrm{d}}} % differential

\newcommand*{\pow}{\ensuremath{\mathcal{P}}}          % power set
\newcommand*{\nset}{\ensuremath{\emptyset}}           % null set
\newcommand*{\setnat}{\ensuremath{\mathbb{N}}}        % set of natural numbers
\newcommand*{\setint}{\ensuremath{\mathbb{Z}}}        % set of integers
\newcommand*{\setrat}{\ensuremath{\mathbb{Q}}}        % set of rational numbers
\newcommand*{\setreal}{\ensuremath{\mathbb{R}}}       % set of real numbers
\newcommand*{\setcomp}{\ensuremath{\mathbb{C}}}       % set of complex numbers
\newcommand*{\posint}{\ensuremath{\setint_{>0}}}      % set of positive integers
\newcommand*{\posreal}{\ensuremath{\setreal_{>0}}}    % set of positive real numbers
\newcommand*{\nonneg}{\ensuremath{\setreal_{\ge 0}}}  % set of nonnegative numbers
\newcommand*{\Nln}[1][n]{\ensuremath{\setnat_{<#1}}}  % set of natural numbers less than n

\newcommand*{\vct}[1]{\ensuremath{\symbf{#1}}}      % vector
\newcommand*{\map}[3]{\ensuremath{#1\colon#2\to#3}} % map
\newcommand*{\grad}{\ensuremath{\nabla}}            % gradient

\newcommand*{\dis}{\ensuremath{\displaystyle}}


\Bdc{Notes on Classical Mechanics}

\section{The Equations of Motion}

The position of a particle in three-dimensional Euclidean space is defined by a vector \(\vct{r} \in \setreal^3\). Its
derivative \(\dis \vct{v} = \frac{\diff \vct{r}}{\diff t}\) with respect to time \(t \in \setreal\) is the
\emph{velocity of the particle}, denoted by a dot above: \(\vct{v} = \dot{\vct{r}}\). The second derivative \(\dis
\ddot{\vct{r}} = \frac{\diff^2 \vct{r}}{\diff t^2}\) is the \emph{acceleration of the particle}.

\subsection{Generalised Coordinates}

In general, the number of independent quantities which are necessary to uniquely define the position of a system is the
number of \emph{degrees of freedom of the system}. Any \(n \in \posint\) quantities \(q_0, \ldots, q_{n - 1}\), each of
which is in some subset of \(\setreal\), which completely define the position of a system with \(n\) degrees of freedom
are referred to as \emph{generalised coordinates of the system}, and the derivatives \(\dot{q}_0, \ldots, \dot{q}_{n
- 1}\) are its \emph{generalised velocities}. Generalised coordinates span the \emph{configuration space of the system}.
We denote generalised coordinates by an \(n\)-dimensional vector \(\vct{q}\).

In principle, if all the coordinates \(\vct{q}\) and velocities \(\dot{\vct{q}}\) of a system are simultaneously
specified for some instant, then accelerations \(\ddot{\vct{q}}\) for that instant are uniquely determined. The
relations between the coordinates, velocities, and accelerations are the \emph{equations of motions of the system},
which are second-order differential equations for the function \(\vct{q}(t)\). Solving for \(\vct{q}(t)\) makes possible
the determination of the motion of the system.

\subsection{The Stationary-Action Principle}

The most general formulation of the law governing the motion of mechanical systems is the \emph{stationary-action
principle} or the \emph{principle of least action}, according to which every mechanical system is characterised by a
definite function \(L\big(\vct{q}(t), \dot{\vct{q}}(t), t\big)\), referred to as the \emph{Lagrangian}, and the motion
of the system is such that a certain condition is satisfied.

Let the system occupy, at the instants \(t_0\) and \(t_1\), positions defined by two sets of values of the coordinates,
\(\vct{q}_0\) and \(\vct{q}_1\). Then the condition is that the system moves between these positions in such a way that
\[
  S[\vct{q}] = \int_{t_0}^{t_1} L\big(\vct{q}(t), \dot{\vct{q}}(t), t\big) \diff t
\]
is stationary. The functional \(S[\vct{q}]\) is referred to as the \emph{action}. The fact that the Lagrangian contains
only \(\vct{q}\) and \(\dot{\vct{q}}\) expresses the aforementioned result that the mechanical state of the system is
determined when the coordinates and velocities are given.

From the calculus of variations, we know that the functional \(S[\vct{q}]\) is stationary if and only if its first
variation vanishes. Thus, the stationary-action principle may also be written
\[
  \updelta S[\vct{q}] = \updelta \int_{t_0}^{t_1} L\big(\vct{q}(t), \dot{\vct{q}}(t), t\big) \diff t = 0.
\]
Per Euler--Lagrange, the above condition is equivalent to \(n\) differential equations
\[
  \frac{\diff}{\diff t} \frac{\partial L}{\partial \dot{q}_i} - \frac{\partial L}{\partial q_i} = 0, \quad i \in \Nln.
\]
The general solution contains \(2 n\) arbitrary constants, which are determined by the initial conditions which specify
the state of the system at some given instant, for example the initial values of all the coordinates and velocities.

The Lagrangian is additive among systems between which interactions may be neglected as the distance between which tends
to infinity, which expresses the fact that the equations of motion of either of the two non-interacting parts cannot
involve quantities pertaining to the other part.

The multiplication of the Lagrangian of a mechanical system by an arbitrary constant also has no effect on the equations
of motion. This corresponds to the natural arbitrariness in the choice of the unit of measurement of the Lagrangian.

As the stationary-action principle is equivalent to the vanishing of the first variation, the Lagrangian is defined only
to within an additive total time derivative of any function or coordinates and time; i.e., two Lagrangians differing by
\(\dis \frac{\diff}{\diff t} f(\vct{q}, t)\) wherein \(f\) is some function of coordinates and time give rise to the
same equations of motion.

\subsection{Galilean Invariance}

It is necessary to choose a \emph{frame of reference} when considering mechanical phenomena, and the equations of motion
are in general different for different frames of reference.

It is found that a frame of reference can always be chosen in which space is homogeneous and isotropic and time is
homogeneous, referred to as an \emph{inertial frame of reference}.

We may draw some immediate inferences concerning the form of the Lagrangian of a particle moving freely in an inertial
frame of reference. The homogeneity of space and time implies that the Lagrangian cannot contain explicitly either
\(\vct{r}\) or \(t\). Thus, \(L\) must be a function of \(\vct{v}\) only. Since space is isotropic, the Lagrangian must
be independent of the direction of \(\vct{v}\) and is therefore a function only of its norm \(\norm{\vct{v}}\). As such,
we see that the Euler--Lagrange equation is
\[
  \frac{\diff}{\diff t} \frac{\partial L}{\partial \vct{v}} = 0,
\]
which implies that \(\vct{v}\) is constant.

Therefore, we conclude that, in an inertial frame of reference, any free motion takes place with a velocity constant in
both magnitude and direction. This is \emph{Newton's first law of motion} or the \emph{principle of inertia}.

Experiments show that another frame moving uniformly in a straight line relative to an inertial frame of reference is
entirely equivalent in all mechanical respects to the inertial frame. Thus, there is an infinity of inertial frames
moving, relative to one another, uniformly in a straight line, and in all these frames the properties of space and time
are identical, and the laws of motion are the same. This is referred to as \emph{Galilean invariance}.

In what follows, unless the contrary is specifically stated, we shall consider only inertial frames of reference.

The coordinates \(\vct{r}_0\) and \(\vct{r}_1\) of a given point in two different frames of reference, of which the
latter moves relative to the former with velocity \(\vct{v}\), are related by
\[
  \vct{r}_0 = \vct{r}_1 + \vct{v} t.
\]
Here, it is assumed that time is the same in the two frames, \(t = t_0 = t_1\), which is one of the foundations of
classical mechanics. These formulae are referred to as a \emph{Galilean transformation}.

\Edc
