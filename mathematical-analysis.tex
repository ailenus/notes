\documentclass[12pt]{article}

\makeatletter
\def\@maketitle {
  \newpage
  \null
  \vskip 2em
  \begin{center}
    \let \footnote \thanks
    {
      \Large\bfseries \MakeUppercase{\@title} \par
    }
    \vskip 1.5em
    {
      \large \lineskip .5em \begin{tabular}[t]{c}
        \MakeUppercase{\@author}
      \end{tabular} \par
    }
    \vskip 1.5em
    {
      \large\scshape \@date
    }
  \end{center}
  \par
  \vskip 1.5em
}
\makeatother

%%%%%%%%%%%%%%%%%%%%%%%%%%%%%%%%%%%%%%%%%%%%%%%%%%%%%%%%%%%%%%%%%%%%%%%%%%%%%%%
%% Packages
%%%%%%%%%%%%%%%%%%%%%%%%%%%%%%%%%%%%%%%%%%%%%%%%%%%%%%%%%%%%%%%%%%%%%%%%%%%%%%%

%% basic
\usepackage[a4paper,margin=1in]{geometry}
\usepackage{fontspec}
\usepackage{polyglossia}

%% formatting
\usepackage{setspace}
\usepackage{enumitem}
\usepackage{tocloft}
\usepackage{titlesec}
\usepackage{fancyhdr}

%% maths
\usepackage{amsmath}
\usepackage{amsthm}
\usepackage{mathtools}
\usepackage[math-style=ISO,bold-style=ISO]{unicode-math}

%% utilities
\usepackage{array}
\usepackage{booktabs}
\usepackage[colorlinks=true]{hyperref}

%%%%%%%%%%%%%%%%%%%%%%%%%%%%%%%%%%%%%%%%%%%%%%%%%%%%%%%%%%%%%%%%%%%%%%%%%%%%%%%
%% Settings
%%%%%%%%%%%%%%%%%%%%%%%%%%%%%%%%%%%%%%%%%%%%%%%%%%%%%%%%%%%%%%%%%%%%%%%%%%%%%%%

%% basic
\setmainlanguage[variant=british]{english}
\SetLanguageKeys{english}{indentfirst=true}
\setmainfont{STIX Two Text}
\setmathfont{STIX Two Math}

%% formatting
\doublespacing
\setlist{noitemsep}
\setlist[1]{labelindent=\parindent}
\setlist[1]{listparindent=\parindent}
\setlist[enumerate,1]{label=(\alph*)}
\setlist[enumerate,2]{label=(\roman*)}
\setlength{\headheight}{15.2pt}
\setlength{\cftsecindent}{0em}
\setlength{\cftsubsecindent}{0em}
\setlength{\cftsubsubsecindent}{0em}
\renewcommand*{\cfttoctitlefont}{\large\scshape}
\renewcommand*{\cftsecfont}{\normalfont}
\renewcommand*{\cftsecpagefont}{\normalfont}
\renewcommand*{\cftsecleader}{\cftdotfill{\cftdotsep}}
\titleformat{\section}{
  \normalfont\Large\scshape\filcenter
}{}{1em}{}
\titleformat{\subsection}{
  \normalfont\large\scshape\raggedright
}{\thesubsection}{1em}{}
\titleformat{\subsubsection}{
  \normalfont\normalsize\scshape\raggedright
}{\thesubsubsection}{1em}{}
\setcounter{footnote}{1}
\DeclareEmphSequence{\bfseries\itshape}

%% maths
\makeatletter
\newtheoremstyle{dfn}
  {\topsep}
  {\topsep}
  {\upshape\addtolength{\@totalleftmargin}{3em}
  \addtolength{\linewidth}{-6em}
  \parshape 1 3em \linewidth}
  {0pt}
  {\scshape}
  {.}
  {5pt plus 1pt minus 1pt}
  {}
\newtheoremstyle{thm}
  {\topsep}
  {\topsep}
  {\slshape\addtolength{\@totalleftmargin}{3em}
  \addtolength{\linewidth}{-6em}
  \parshape 1 3em \linewidth}
  {0pt}
  {\bfseries}
  {.}
  {5pt plus 1pt minus 1pt}
  {}
\newtheoremstyle{xmp}
  {\topsep}
  {\topsep}
  {\upshape\addtolength{\@totalleftmargin}{3em}
  \addtolength{\linewidth}{-6em}
  \parshape 1 3em \linewidth}
  {0pt}
  {\bfseries}
  {.}
  {5pt plus 1pt minus 1pt}
  {}
\newtheoremstyle{xrc}
  {\topsep}
  {\topsep}
  {\upshape\addtolength{\@totalleftmargin}{3em}
  \addtolength{\linewidth}{-6em}
  \parshape 1 3em \linewidth}
  {0pt}
  {\bfseries\scshape}
  {.}
  {5pt plus 1pt minus 1pt}
  {}
\newtheoremstyle{sol}
  {\topsep}
  {\topsep}
  {\upshape}
  {0pt}
  {\itshape}
  {.}
  {5pt plus 1pt minus 1pt}
  {}
\makeatother
\theoremstyle{dfn}
\newtheorem{dfn}{Definition}
\theoremstyle{thm}
\newtheorem{thm}{Theorem}
\newtheorem{lem}{Lemma}
\newtheorem{cor}{Corollary}
\renewcommand\qedsymbol{\(\mdlgwhtsquare\)}
\theoremstyle{xmp}
\newtheorem{xmp}{Example}
\theoremstyle{xrc}
\newtheorem{xrc}{Exercise}
\theoremstyle{sol}
\newtheorem*{XxmpX}{Solution}
\newenvironment{solution}{
  \renewcommand{\qedsymbol}{\(\mdlgwhtlozenge\)}
  \pushQED{\qed}\begin{XxmpX}
}{\popQED\end{XxmpX}}
\def\equationautorefname~#1\null{(#1)\null}
\newcommand{\dfnautorefname}{definition}
\newcommand{\thmautorefname}{theorem}
\newcommand{\lemautorefname}{lemma}
\newcommand{\corautorefname}{corollary}

%% aliases
\newcommand*{\Bdc}[2][\today]{
  \title{#2}
  \author{Yannan Mao}
  \date{#1}
  \lhead{\scshape #2}
  \rhead{\scshape Yannan Mao}
  \begin{document}
  \maketitle
  \tableofcontents
  \thispagestyle{empty}
  \clearpage
  \pagestyle{fancy}
  \setcounter{page}{1}
}
\newcommand*{\Edc}{\end{document}}
\newcommand*{\Bdf}{\begin{dfn}}
\newcommand*{\Edf}{\end{dfn}}
\newcommand*{\Bth}{\begin{thm}}
\newcommand*{\Eth}{\end{thm}}
\newcommand*{\Blm}{\begin{lem}}
\newcommand*{\Elm}{\end{lem}}
\newcommand*{\Bcr}{\begin{cor}}
\newcommand*{\Ecr}{\end{cor}}
\newcommand*{\Bxm}{\begin{xmp}}
\newcommand*{\Exm}{\end{xmp}}
\newcommand*{\Bxr}{\begin{xrc}}
\newcommand*{\Exr}{\end{xrc}}
\newcommand*{\Bpr}{\begin{proof}}
\newcommand*{\Epr}{\end{proof}}
\newcommand*{\Bsl}{\begin{solution}}
\newcommand*{\Esl}{\end{solution}}

\DeclareMathOperator{\dom}{dom}           % domain
\DeclareMathOperator{\im}{im}             % image
\DeclareMathOperator*{\argmax}{arg\,max}  % arg max
\DeclareMathOperator*{\argmin}{arg\,min}  % arg min

\DeclarePairedDelimiter{\abs}{\lvert}{\rvert}               % absolute value, cardinality
\DeclarePairedDelimiter{\norm}{\lVert}{\rVert}              % norm
\DeclarePairedDelimiterX{\inn}[2]{\langle}{\rangle}{#1,#2}  % inner product
\DeclarePairedDelimiter{\set}{\{}{\}}                       % set

\newcommand*{\ee}{\ensuremath{\mathrm{e}}}              % e
\newcommand*{\ii}{\ensuremath{\mathrm{i}}}              % i
\newcommand*{\diff}{\ensuremath{\mathop{}\!\mathrm{d}}} % differential

\newcommand*{\pow}{\ensuremath{\mathcal{P}}}          % power set
\newcommand*{\nset}{\ensuremath{\emptyset}}           % null set
\newcommand*{\setnat}{\ensuremath{\mathbb{N}}}        % set of natural numbers
\newcommand*{\setint}{\ensuremath{\mathbb{Z}}}        % set of integers
\newcommand*{\setrat}{\ensuremath{\mathbb{Q}}}        % set of rational numbers
\newcommand*{\setreal}{\ensuremath{\mathbb{R}}}       % set of real numbers
\newcommand*{\setcomp}{\ensuremath{\mathbb{C}}}       % set of complex numbers
\newcommand*{\posint}{\ensuremath{\setint_{>0}}}      % set of positive integers
\newcommand*{\posreal}{\ensuremath{\setreal_{>0}}}    % set of positive real numbers
\newcommand*{\nonneg}{\ensuremath{\setreal_{\ge 0}}}  % set of nonnegative numbers
\newcommand*{\Nln}[1][n]{\ensuremath{\setnat_{<#1}}}  % set of natural numbers less than n

\newcommand*{\vct}[1]{\ensuremath{\symbf{#1}}}      % vector
\newcommand*{\map}[3]{\ensuremath{#1\colon#2\to#3}} % map
\newcommand*{\grad}{\ensuremath{\nabla}}            % gradient

\newcommand*{\dis}{\ensuremath{\displaystyle}}


\Bdc{Mathematical Analysis}

\section{Calculus of Variations}

\subsection{Linear Forms}

Let \(F\) be a field, and let \(V\) be a vector space over \(F\). A linear map from \(V\) into \(F\) is referred to as a
\emph{linear form on \(V\)}. Equivalently, a function \(\map{f}{V}{F}\) is a linear form if \(f(\lambda \vct{a}
+ \vct{b}) = \lambda f(\vct{a}) + f(\vct{b})\) for any \(\lambda \in F\) and any \(\vct{a}, \vct{b} \in V\). Linear
forms are also known as \emph{linear functionals}.

Let \([x_0, x_1]\) be a closed interval on \(\setreal\), and let \(C^0\big([x_0, x_1]\big)\) be the vector space of
continuous real functions on \([x_0, x_1]\). Then \(\map{J}{C^0\big([x_0, x_1]\big)}{\setreal}\) defined by
\[
  J(f) = \int_{x_0}^{x_1} f(x) \diff x
\]
is a linear form on \(C^0\big([x_0, x_1]\big)\).

\subsection{Functionals and Their Extrema}

Let \([x_0, x_1]\) be a closed interval on \(\setreal\), and let \(C^2\big([x_0, x_1]\big)\) be the set of twice
continuously differentiable real functions on \([x_0, x_1]\). We refer to linear forms on \(C^2\big([x_0, x_1]\big)\) as
\emph{functionals}. We denote a functional by enclosing its variable in square brackets.

Let \(\Omega \subseteq C^2\big([x_0, x_1]\big)\) be a set of functions. A functional \(\map{J}{\Omega}{\setreal}\) is
said to obtain an \emph{extremum at function \(f\)} if there exists an \(\varepsilon \in \posreal\) such that \(J[g]
- J[f]\) has the same sign for any \(g \in \Omega\) which satisfies \(\forall x \in [x_0, x_1] \, \big(\abs{g(x) - f(x)}
< \varepsilon\big)\).

\subsection{Variations}

Suppose \(g \in \Omega\) is a function whereat the functional \(J\) obtains an extremum. Take another function
\(\map{\eta}{[x_0, x_1]}{\setreal}\) which vanishes at \(x_0\) and \(x_1\). We then form the family of functions
\[
  \varphi(x, \varepsilon) = g(x) + \varepsilon \eta(x)
\]
with \(\varepsilon \in \setreal\). Note that with any given \(\varepsilon\) we have \(\varphi \in \Omega\).

We see that
\[
  \eta(x) = \frac{\partial \varphi}{\partial \varepsilon}
\]
and so we refer to \(\varepsilon \eta(x)\) as a \emph{variation of \(g\)} and denote it by \(\updelta g\).

Let \(\psi(\varepsilon) = J[g + \varepsilon \eta]\) be a function of \(\varepsilon\). The postulate that \(g\) shall
give an extremum of \{J\} implies that \(\psi\) shall possess a minimum for \(\varepsilon = 0\), so as a necessary
condition we have the equation
\[
  \psi'(0) = 0.
\]

\subsection{Euler--Lagrange Equation}

Let \(\Omega \subset C^2\big([x_0, x_1]\big)\) be given by \(\Omega = \set*{f \in C^2\big([x_0, x_1]\big) : y_0 = f(x_0)
\, \land \, y_1 = f(x_1)}\) wherein \(y_0, y_1 \in \setreal\) are prescribed. Consider a functional of the form
\begin{align}
  \label{eq0}
  J(f) = \int_{x_0}^{x_1} L\big(x, f'(x), f(x)\big) \diff x
\end{align}
wherein \(L\) is a twice continuously differentiable function with respect to \(x\), \(f'\), and \(f\).

\Bth[Euler--Lagrange equation]
  The functional \(J\) defined in \eqref{eq0} obtains an extremum at function \(f\) if and only if
  \[
    \frac{\partial L}{\partial f} - \frac{\diff}{\diff x} \frac{\partial L}{\partial f'}.
  \]
\Eth

\Edc
