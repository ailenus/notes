\documentclass[12pt]{article}

\makeatletter
\def\@maketitle {
    \newpage
    \null
    \vskip 2em
    \begin{center}
        \let \footnote \thanks
        {
            \Large\bfseries \MakeUppercase{\@title} \par
        }
        \vskip 1.5em
        {
            \large \lineskip .5em \begin{tabular}[t]{c}
                \MakeUppercase{\@author}
            \end{tabular} \par
        }
        \vskip 1.5em
        {
            \large\scshape \@date
        }
    \end{center}
    \par
    \vskip 1.5em
}
\makeatother

%%%%%%%%%%%%%%%%%%%%%%%%%%%%%%%%%%%%%%%%%%%%%%%%%%%%%%%%%%%%%%%%%%%%%%%%%%%%%%%
%% Packages
%%%%%%%%%%%%%%%%%%%%%%%%%%%%%%%%%%%%%%%%%%%%%%%%%%%%%%%%%%%%%%%%%%%%%%%%%%%%%%%

%% basic
\usepackage[a4paper,margin=1in]{geometry}
\usepackage{fontspec}
\usepackage{polyglossia}

%% formatting
\usepackage{setspace}
\usepackage{enumitem}
\usepackage{tocloft}
\usepackage{titlesec}
\usepackage{fancyhdr}

%% maths
\usepackage{amsmath}
\usepackage{amsthm}
\usepackage{mathtools}
\usepackage[math-style=ISO,bold-style=ISO]{unicode-math}

%% utilities
\usepackage{array}
\usepackage{booktabs}
\usepackage[colorlinks=true]{hyperref}

%%%%%%%%%%%%%%%%%%%%%%%%%%%%%%%%%%%%%%%%%%%%%%%%%%%%%%%%%%%%%%%%%%%%%%%%%%%%%%%
%% Settings
%%%%%%%%%%%%%%%%%%%%%%%%%%%%%%%%%%%%%%%%%%%%%%%%%%%%%%%%%%%%%%%%%%%%%%%%%%%%%%%

%% basic
\setmainlanguage[variant=british]{english}
\SetLanguageKeys{english}{indentfirst=true}
\setmainfont{STIX Two Text}
\setmathfont{STIX Two Math}

%% formatting
\doublespacing
\setlist{noitemsep}
\setlist[1]{labelindent=\parindent}
\setlist[1]{listparindent=\parindent}
\setlist[enumerate,1]{label=(\alph*)}
\setlist[enumerate,2]{label=(\roman*)}
\setlength{\headheight}{15.2pt}
\setlength{\cftsecindent}{0em}
\setlength{\cftsubsecindent}{0em}
\setlength{\cftsubsubsecindent}{0em}
\renewcommand*{\cfttoctitlefont}{\large\scshape}
\renewcommand*{\cftsecfont}{\normalfont}
\renewcommand*{\cftsecpagefont}{\normalfont}
\renewcommand*{\cftsecleader}{\cftdotfill{\cftdotsep}}
\titleformat{\section}{
    \normalfont\Large\scshape\filcenter
}{}{1em}{}
\titleformat{\subsection}{
    \normalfont\large\scshape\raggedright
}{\thesubsection}{1em}{}
\titleformat{\subsubsection}{
    \normalfont\normalsize\scshape\raggedright
}{\thesubsubsection}{1em}{}
\setcounter{footnote}{1}

%% maths
\makeatletter
\newtheoremstyle{dfn}
    {\topsep}
    {\topsep}
    {\upshape\addtolength{\@totalleftmargin}{3em}
    \addtolength{\linewidth}{-6em}
    \parshape 1 3em \linewidth}
    {0pt}
    {\scshape}
    {.}
    {5pt plus 1pt minus 1pt}
    {}
\newtheoremstyle{thm}
    {\topsep}
    {\topsep}
    {\slshape\addtolength{\@totalleftmargin}{3em}
    \addtolength{\linewidth}{-6em}
    \parshape 1 3em \linewidth}
    {0pt}
    {\bfseries}
    {.}
    {5pt plus 1pt minus 1pt}
    {}
\newtheoremstyle{xrc}
    {\topsep}
    {\topsep}
    {\upshape\addtolength{\@totalleftmargin}{3em}
    \addtolength{\linewidth}{-6em}
    \parshape 1 3em \linewidth}
    {0pt}
    {\bfseries\scshape}
    {.}
    {5pt plus 1pt minus 1pt}
    {}
\newtheoremstyle{sol}
    {\topsep}
    {\topsep}
    {\upshape}
    {0pt}
    {\itshape}
    {.}
    {5pt plus 1pt minus 1pt}
    {}
\makeatother
\theoremstyle{dfn}
\newtheorem{dfn}{Definition}
\theoremstyle{thm}
\newtheorem{thm}{Theorem}
\newtheorem{lem}{Lemma}
\newtheorem{cor}{Corollary}
\renewcommand\qedsymbol{\(\mdlgwhtsquare\)}
\theoremstyle{xrc}
\newtheorem{xrc}{Exercise}
\theoremstyle{sol}
\newtheorem*{XxmpX}{Solution}
\newenvironment{solution}{
    \renewcommand{\qedsymbol}{\(\mdlgwhtlozenge\)}
    \pushQED{\qed}\begin{XxmpX}
}{\popQED\end{XxmpX}}
\def\equationautorefname~#1\null{(#1)\null}
\newcommand{\dfnautorefname}{definition}
\newcommand{\thmautorefname}{theorem}
\newcommand{\lemautorefname}{lemma}
\newcommand{\corautorefname}{corollary}

%% aliases
\newcommand*{\Bdc}[2][\today]{
    \title{#2}
    \author{Yannan Mao}
    \date{#1}
    \lhead{\scshape #2}
    \rhead{\scshape Yannan Mao}
    \begin{document}
    \maketitle
    \tableofcontents
    \thispagestyle{empty}
    \clearpage
    \pagestyle{fancy}
    \setcounter{page}{1}
}
\newcommand*{\Edc}{\end{document}}
\newcommand*{\Bdf}{\begin{dfn}}
\newcommand*{\Edf}{\end{dfn}}
\newcommand*{\Bth}{\begin{thm}}
\newcommand*{\Eth}{\end{thm}}
\newcommand*{\Blm}{\begin{lem}}
\newcommand*{\Elm}{\end{lem}}
\newcommand*{\Bcr}{\begin{cor}}
\newcommand*{\Ecr}{\end{cor}}
\newcommand*{\Bxr}{\begin{xrc}}
\newcommand*{\Exr}{\end{xrc}}
\newcommand*{\Bpr}{\begin{proof}}
\newcommand*{\Epr}{\end{proof}}
\newcommand*{\Bsl}{\begin{solution}}
\newcommand*{\Esl}{\end{solution}}

\DeclareMathOperator{\dom}{dom}             % domain
\DeclareMathOperator{\im}{im}               % image
\DeclareMathOperator*{\argmax}{arg\,max}    % arg max
\DeclareMathOperator*{\argmin}{arg\,min}    % arg min

\DeclarePairedDelimiter{\abs}{\lvert}{\rvert}               % absolute value, cardinality
\DeclarePairedDelimiter{\norm}{\lVert}{\rVert}              % norm
\DeclarePairedDelimiterX{\inn}[2]{\langle}{\rangle}{#1,#2}  % inner product
\DeclarePairedDelimiter{\set}{\{}{\}}                       % set

\newcommand*{\ee}{\ensuremath{\mathrm{e}}}              % e
\newcommand*{\ii}{\ensuremath{\mathrm{i}}}              % i
\newcommand*{\diff}{\ensuremath{\mathop{}\!\mathrm{d}}} % differential

\newcommand*{\pow}{\ensuremath{\mathcal{P}}}            % power set
\newcommand*{\nset}{\ensuremath{\emptyset}}             % null set
\newcommand*{\setnat}{\ensuremath{\mathbb{N}}}          % set of natural numbers
\newcommand*{\setint}{\ensuremath{\mathbb{Z}}}          % set of integers
\newcommand*{\setrat}{\ensuremath{\mathbb{Q}}}          % set of rational numbers
\newcommand*{\setreal}{\ensuremath{\mathbb{R}}}         % set of real numbers
\newcommand*{\setcomp}{\ensuremath{\mathbb{C}}}         % set of complex numbers
\newcommand*{\posint}{\ensuremath{\setint_{>0}}}        % set of positive integers
\newcommand*{\posreal}{\ensuremath{\setreal_{>0}}}      % set of positive real numbers
\newcommand*{\nonneg}{\ensuremath{\setreal_{\ge 0}}}    % set of nonnegative numbers
\newcommand*{\Nln}[1][n]{\ensuremath{\setnat_{<#1}}}    % set of natural numbers less than n

\newcommand*{\vct}[1]{\ensuremath{\symbf{#1}}}      % vector
\newcommand*{\map}[3]{\ensuremath{#1\colon#2\to#3}} % map
\newcommand*{\grad}{\ensuremath{\nabla}}            % gradient

\newcommand*{\dis}{\ensuremath{\displaystyle}}


\Bdc{Notes on Mathematical Analysis}

\section{The Calculus of Variations}

\subsection{Linear Forms}

Let \(F\) be a field, and let \(V\) be a vector space over \(F\). A linear map from \(V\) into \(F\) is referred to as a
\emph{linear form on \(V\)}. Equivalently, a function \(\map{f}{V}{F}\) is a linear form if \(f(\lambda \vct{a}
+ \vct{b}) = \lambda f(\vct{a}) + f(\vct{b})\) for any \(\lambda \in F\) and any \(\vct{a}, \vct{b} \in V\). Linear
forms are also known as \emph{linear functionals}.

Let \([x_0, x_1]\) be a closed interval on \(\setreal\), and let \(C^0\big([x_0, x_1]\big)\) be the vector space of
continuous real functions on \([x_0, x_1]\). Then \(\map{J}{C^0\big([x_0, x_1]\big)}{\setreal}\) defined by
\[
  J(f) = \int_{x_0}^{x_1} f(x) \diff x
\]
is a linear form on \(C^0\big([x_0, x_1]\big)\).

\subsection{Functionals and Their Extrema}

Let \([x_0, x_1]\) be a closed interval on \(\setreal\), and let \(C^2\big([x_0, x_1]\big)\) be the set of twice
continuously differentiable real functions on \([x_0, x_1]\). We refer to linear forms on \(C^2\big([x_0, x_1]\big)\) as
\emph{functionals}. We denote a functional by enclosing its variable in square brackets.

Let \(\Omega \subseteq C^2\big([x_0, x_1]\big)\) be a set of functions. A functional \(\map{J}{\Omega}{\setreal}\) is
said to obtain an \emph{extremum at function \(f\)} if there exists an \(\varepsilon \in \posreal\) such that \(J[g]
- J[f]\) has the same sign for any \(g \in \Omega\) which satisfies \(\forall x \in [x_0, x_1] \, \big(\abs{g(x) - f(x)}
< \varepsilon\big)\).

\subsection{Vanishing of the First Variation}

Let \(\Omega \subset C^2\big([x_0, x_1]\big)\) be given by \(\Omega = \set*{f \in C^2\big([x_0, x_1]\big) : y_0 = f(x_0)
\, \land \, y_1 = f(x_1)}\) wherein \(y_0, y_1 \in \setreal\) are prescribed. Consider a functional of the form
\begin{align}
  \label{eq0}
  J[f] = \int_{x_0}^{x_1} L\big(x, f'(x), f(x)\big) \diff x
\end{align}
wherein \(L\) is a twice continuously differentiable function with respect to \(x\), \(f'\), and \(f\).

Suppose \(f \in \Omega\) is a function whereat the functional \(J\) obtains an extremum. Take another function
\(\eta \in C^2\big([x_0, x_1]\big)\) which vanishes at \(x_0\) and \(x_1\). We then form the family of functions
\[
  \varphi(x, \varepsilon) = f(x) + \varepsilon \eta(x)
\]
with \(\varepsilon \in \setreal\). Note that with any given \(\varepsilon\) we have \(\varphi \in \Omega\).

We see that
\[
  \eta(x) = \frac{\partial \varphi}{\partial \varepsilon}
\]
and so we refer to \(\varepsilon \eta(x)\) as a \emph{variation of \(f\)} and denote it by \(\updelta f\).

Let \(\psi(\varepsilon) = J[f + \varepsilon \eta]\) be a real function. The postulate that \(f\) shall give an extremum
of \(J\) implies that \(\psi\) shall possess an extremum for \(\varepsilon = 0\), and so it is necessary that
\[
  \psi'(0) = 0.
\]
If \(f\) satisfies \(\psi'(0) = 0\) for any \(\eta\), we then say that \(J\) is \emph{stationary at \(f\)}.

In general, we refer to \(\varepsilon \psi'(0)\) as the \emph{first variation of \(J\)} and denote it by \(\updelta J\).
Thus, the stationary character of \(J\) at \(f\) is equivalent to the vanishing of the first variation.

\subsection{The Fundamental Lemma of the Calculus of Variations}

\Blm[fundamental lemma of the calculus of variations]
  If a function \(f \in C^0\big([x_0, x_1]\big)\) satisfies
  \[
    \int_{x_0}^{x_1} \eta(x) f(x) \diff x = 0
  \]
  for any \(\eta \in C^2\big([x_0, x_1]\big)\) such that \(\eta(x_0) = \eta(x_1) = 0\), then \(f(x) = 0\) for any
  \(x \in [x_0, x_1]\).
\Elm
\Bpr
  We assume, for the sake of contradiction, that there exists a \(\xi \in [x_0, x_1]\) such that \(f(\xi) > 0\) and
  which satisfies all the prescribed conditions. Then, as \(f\) is continuous on \([x_0, x_1]\), there exists an
  \(\alpha \in \posreal\) such that \(f(x) > 0\) for any \(x \in [\xi - \alpha, \xi + \alpha] \subseteq [x_0, x_1]\).
  Let \(\eta \in C^2\big([x_0, x_1]\big)\) be defined by
  \[
    \eta(x) = \begin{cases}
      \big((x - \xi)^2 - \alpha^2\big)^4 & \text{if } x \in [\xi - \alpha, \xi + \alpha],\\
      0 & \text{otherwise}.
    \end{cases}
  \]
  We see that \(\eta(x) f(x) > 0\) for any \(x \in [\xi - \alpha, \xi + \alpha]\) and that \(\eta(x) f(x) = 0\) for any
  \(x \in [x_0, \xi - \alpha) \cup (\xi + \alpha, x_1]\). It follows that
  \[
    \int_{x_0}^{x_1} \eta(x) f(x) \diff x > 0,
  \]
  which is a contradiction. Therefore, \(f(\xi)\) cannot be positive. For the same reasons, \(f(\xi)\) cannot be
  negative. Hence, \(f(x)\) must vanish for any \(x \in [x_0, x_1]\).

  Thus, our original proposition holds.
\Epr

\subsection{The Euler--Lagrange Equation}

\Bth[Euler--Lagrange]
  The functional \(J\) defined in \eqref{eq0} is stationary at function \(f\) if and only if
  \[
    \frac{\partial L}{\partial f} - \frac{\diff}{\diff x} \frac{\partial L}{\partial f'}.
  \]
\Eth

\Edc
