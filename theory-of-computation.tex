\documentclass[12pt]{article}

\makeatletter
\def\@maketitle {
    \newpage
    \null
    \vskip 2em
    \begin{center}
        \let \footnote \thanks
        {
            \Large\bfseries \MakeUppercase{\@title} \par
        }
        \vskip 1.5em
        {
            \large \lineskip .5em \begin{tabular}[t]{c}
                \MakeUppercase{\@author}
            \end{tabular} \par
        }
        \vskip 1.5em
        {
            \large\scshape \@date
        }
    \end{center}
    \par
    \vskip 1.5em
}
\makeatother

%%%%%%%%%%%%%%%%%%%%%%%%%%%%%%%%%%%%%%%%%%%%%%%%%%%%%%%%%%%%%%%%%%%%%%%%%%%%%%%
%% Packages
%%%%%%%%%%%%%%%%%%%%%%%%%%%%%%%%%%%%%%%%%%%%%%%%%%%%%%%%%%%%%%%%%%%%%%%%%%%%%%%

%% basic
\usepackage[a4paper,margin=1in]{geometry}
\usepackage{fontspec}
\usepackage{polyglossia}

%% formatting
\usepackage{setspace}
\usepackage{enumitem}
\usepackage{tocloft}
\usepackage{titlesec}
\usepackage{fancyhdr}

%% maths
\usepackage{amsmath}
\usepackage{amsthm}
\usepackage{mathtools}
\usepackage[math-style=ISO,bold-style=ISO]{unicode-math}

%% utilities
\usepackage{array}
\usepackage{booktabs}
\usepackage[colorlinks=true]{hyperref}

%%%%%%%%%%%%%%%%%%%%%%%%%%%%%%%%%%%%%%%%%%%%%%%%%%%%%%%%%%%%%%%%%%%%%%%%%%%%%%%
%% Settings
%%%%%%%%%%%%%%%%%%%%%%%%%%%%%%%%%%%%%%%%%%%%%%%%%%%%%%%%%%%%%%%%%%%%%%%%%%%%%%%

%% basic
\setmainlanguage[variant=british]{english}
\SetLanguageKeys{english}{indentfirst=true}
\setmainfont{STIX Two Text}
\setmathfont{STIX Two Math}

%% formatting
\doublespacing
\setlist{noitemsep}
\setlist[1]{labelindent=\parindent}
\setlist[1]{listparindent=\parindent}
\setlist[enumerate,1]{label=(\alph*)}
\setlist[enumerate,2]{label=(\roman*)}
\setlength{\headheight}{15.2pt}
\setlength{\cftsecindent}{0em}
\setlength{\cftsubsecindent}{0em}
\setlength{\cftsubsubsecindent}{0em}
\renewcommand*{\cfttoctitlefont}{\large\scshape}
\renewcommand*{\cftsecfont}{\normalfont}
\renewcommand*{\cftsecpagefont}{\normalfont}
\renewcommand*{\cftsecleader}{\cftdotfill{\cftdotsep}}
\titleformat{\section}{
    \normalfont\Large\scshape\filcenter
}{}{1em}{}
\titleformat{\subsection}{
    \normalfont\large\scshape\raggedright
}{\thesubsection}{1em}{}
\titleformat{\subsubsection}{
    \normalfont\normalsize\scshape\raggedright
}{\thesubsubsection}{1em}{}
\setcounter{footnote}{1}

%% maths
\makeatletter
\newtheoremstyle{dfn}
    {\topsep}
    {\topsep}
    {\upshape\addtolength{\@totalleftmargin}{3em}
    \addtolength{\linewidth}{-6em}
    \parshape 1 3em \linewidth}
    {0pt}
    {\scshape}
    {.}
    {5pt plus 1pt minus 1pt}
    {}
\newtheoremstyle{thm}
    {\topsep}
    {\topsep}
    {\slshape\addtolength{\@totalleftmargin}{3em}
    \addtolength{\linewidth}{-6em}
    \parshape 1 3em \linewidth}
    {0pt}
    {\bfseries}
    {.}
    {5pt plus 1pt minus 1pt}
    {}
\newtheoremstyle{xrc}
    {\topsep}
    {\topsep}
    {\upshape\addtolength{\@totalleftmargin}{3em}
    \addtolength{\linewidth}{-6em}
    \parshape 1 3em \linewidth}
    {0pt}
    {\bfseries\scshape}
    {.}
    {5pt plus 1pt minus 1pt}
    {}
\newtheoremstyle{sol}
    {\topsep}
    {\topsep}
    {\upshape}
    {0pt}
    {\itshape}
    {.}
    {5pt plus 1pt minus 1pt}
    {}
\makeatother
\theoremstyle{dfn}
\newtheorem{dfn}{Definition}
\theoremstyle{thm}
\newtheorem{thm}{Theorem}
\newtheorem{lem}{Lemma}
\newtheorem{cor}{Corollary}
\renewcommand\qedsymbol{\(\mdlgwhtsquare\)}
\theoremstyle{xrc}
\newtheorem{xrc}{Exercise}
\theoremstyle{sol}
\newtheorem*{XxmpX}{Solution}
\newenvironment{solution}{
    \renewcommand{\qedsymbol}{\(\mdlgwhtlozenge\)}
    \pushQED{\qed}\begin{XxmpX}
}{\popQED\end{XxmpX}}
\def\equationautorefname~#1\null{(#1)\null}
\newcommand{\dfnautorefname}{definition}
\newcommand{\thmautorefname}{theorem}
\newcommand{\lemautorefname}{lemma}
\newcommand{\corautorefname}{corollary}

%% aliases
\newcommand*{\Bdc}[2][\today]{
    \title{#2}
    \author{Yannan Mao}
    \date{#1}
    \lhead{\scshape #2}
    \rhead{\scshape Yannan Mao}
    \begin{document}
    \maketitle
    \tableofcontents
    \thispagestyle{empty}
    \clearpage
    \pagestyle{fancy}
    \setcounter{page}{1}
}
\newcommand*{\Edc}{\end{document}}
\newcommand*{\Bdf}{\begin{dfn}}
\newcommand*{\Edf}{\end{dfn}}
\newcommand*{\Bth}{\begin{thm}}
\newcommand*{\Eth}{\end{thm}}
\newcommand*{\Blm}{\begin{lem}}
\newcommand*{\Elm}{\end{lem}}
\newcommand*{\Bcr}{\begin{cor}}
\newcommand*{\Ecr}{\end{cor}}
\newcommand*{\Bxr}{\begin{xrc}}
\newcommand*{\Exr}{\end{xrc}}
\newcommand*{\Bpr}{\begin{proof}}
\newcommand*{\Epr}{\end{proof}}
\newcommand*{\Bsl}{\begin{solution}}
\newcommand*{\Esl}{\end{solution}}

\DeclareMathOperator{\dom}{dom}             % domain
\DeclareMathOperator{\im}{im}               % image
\DeclareMathOperator*{\argmax}{arg\,max}    % arg max
\DeclareMathOperator*{\argmin}{arg\,min}    % arg min

\DeclarePairedDelimiter{\abs}{\lvert}{\rvert}               % absolute value, cardinality
\DeclarePairedDelimiter{\norm}{\lVert}{\rVert}              % norm
\DeclarePairedDelimiterX{\inn}[2]{\langle}{\rangle}{#1,#2}  % inner product
\DeclarePairedDelimiter{\set}{\{}{\}}                       % set

\newcommand*{\ee}{\ensuremath{\mathrm{e}}}              % e
\newcommand*{\ii}{\ensuremath{\mathrm{i}}}              % i
\newcommand*{\diff}{\ensuremath{\mathop{}\!\mathrm{d}}} % differential

\newcommand*{\pow}{\ensuremath{\mathcal{P}}}            % power set
\newcommand*{\nset}{\ensuremath{\emptyset}}             % null set
\newcommand*{\setnat}{\ensuremath{\mathbb{N}}}          % set of natural numbers
\newcommand*{\setint}{\ensuremath{\mathbb{Z}}}          % set of integers
\newcommand*{\setrat}{\ensuremath{\mathbb{Q}}}          % set of rational numbers
\newcommand*{\setreal}{\ensuremath{\mathbb{R}}}         % set of real numbers
\newcommand*{\setcomp}{\ensuremath{\mathbb{C}}}         % set of complex numbers
\newcommand*{\posint}{\ensuremath{\setint_{>0}}}        % set of positive integers
\newcommand*{\posreal}{\ensuremath{\setreal_{>0}}}      % set of positive real numbers
\newcommand*{\nonneg}{\ensuremath{\setreal_{\ge 0}}}    % set of nonnegative numbers
\newcommand*{\Nln}[1][n]{\ensuremath{\setnat_{<#1}}}    % set of natural numbers less than n

\newcommand*{\vct}[1]{\ensuremath{\symbf{#1}}}      % vector
\newcommand*{\map}[3]{\ensuremath{#1\colon#2\to#3}} % map
\newcommand*{\grad}{\ensuremath{\nabla}}            % gradient

\newcommand*{\dis}{\ensuremath{\displaystyle}}


\usepackage{tikz}
\usetikzlibrary{automata,decorations.markings}
\newcommand*{\pt}{5mm}
\newcommand*{\lemautorefname}{Lemma}
\newcommand*{\thmautorefname}{Theorem}

\Bdc{Notes on the Theory of Computation}

\section{Automata and Formal Languages}

An \emph{alphabet} is a finite set \(\Sigma\), and a \emph{word over the alphabet \(\Sigma\)} is a finite sequence of
the elements of \(\Sigma\). If a word \(w\) is the sequence \((w_0, \ldots, w_n)\) for some \(n \in \setnat\), we may
write \(w\) as the concatenation \(w_0 \cdots w_n\). If \(w = a \cdots a\) wherein \(a\) is repeated \(n\) times for
some \(n \in \posint\), we may write \(w\) as \(a^n\). The empty word is denoted by \(\epsilon\), and for any element
\(a\) of an alphabet \(a^0\) is the empty word. The set of all words over \(\Sigma\) is \(\Sigma^*\)\footnote{\(^*\)
denotes the unary operator of Kleene star, defined as \(A^* = \set{a_0 \cdots a_n : n \in \setnat
\land \forall \, i \in \Nln[n + 1] \, (a_i \in A)} \cup \set{\epsilon}\) for a subset \(A\) of an alphabet, and
\(a^* = \set{a^n : n \in \setnat}\) for an element \(a\) of an alphabet.}. A \emph{formal language over the alphabet
\(\Sigma\)} is a subset of \(\Sigma^*\).  The attributive ``formal'' connotes that such languages lack semantics.

An \emph{automaton} is an ordered sequence that \emph{accepts} some words over an
alphabet. The set of words an automaton accepts forms a language, which is
unique, in which case we say the automaton \emph{recognises} the language. Given
an automaton \(M\), we may speak of the unique language recognised by \(M\) as
the \emph{language of the automaton \(M\)}. An automaton may accept no word, in
which case the language thereof is \(\nset\). Two automata are equivalent if
they recognise the same language.

\subsection{Finite-State Automata and Regular Languages}

\subsubsection{Deterministic Finite-State Automata}

\Bdf
    A \emph{deterministic finite-state automaton} is an ordered quintuple
    \((\Sigma, Q, \delta, q_0, F)\) wherein
    \begin{enumerate}
        \item \(\Sigma\) is an alphabet,
        \item \(Q\) is a finite set of \emph{states},
        \item \(\map{\delta}{Q \times \Sigma}{Q}\) is the \emph{transition
        function},
        \item \(q_0 \in Q\) is the \emph{initial state}, and
        \item \(F \subseteq Q\) is the set of \emph{accepting states}.
    \end{enumerate}
\Edf

Let \(M = (\Sigma, Q, \delta, q_0, F)\) be a deterministic finite-state
automaton and let \(w = w_0 \cdots w_n\) wherein \(n \in \setnat\) be a word
over \(\Sigma\).  Then \(M\) accepts \(w\) if there exists a sequence of states
\((r_0, \ldots, r_{n + 1})\) in \(Q\) such that
\begin{enumerate}
    \item \(r_0 = q_0\),
    \item \(\delta(r_i, w_i) = r_{i + 1}\) for \(i \in \Nln[n + 1]\), and
    \item \(r_{n + 1} \in F\).
\end{enumerate}
Furthermore, \(M\) accepts \(\epsilon\) if \(q_0 \in F\).

\subsubsection{Nondeterministic Finite-State Automata}

\Bdf
    A \emph{nondeterministic finite-state automaton} is an ordered quintuple
    \((\Sigma, Q, \delta, q_0, F)\) wherein
    \begin{enumerate}
        \item \(\Sigma\) is an alphabet,
        \item \(Q\) is a finite set of states,
        \item \(\map{\delta}{Q \times (\Sigma \cup \set{\epsilon})}{\pow(Q)}\)
        is the transition function,
        \item \(q_0 \in Q\) is the initial state, and
        \item \(F \subseteq Q\) is the set of accepting states.
    \end{enumerate}
\Edf

Let \(N = (\Sigma, Q, \delta, q_0, F)\) be a nondeterministic finite-state
automaton and let \(w\) be a word over \(\Sigma\). Then \(N\) accepts \(w\) if
\(w = w_0 \cdots w_n\) wherein \(n \in \setnat\) such that each \(w_i \in \Sigma
\cup \set{\epsilon}\) for some \(i \in \Nln[n + 1]\) and that there exists a
sequence of states \((r_0, \ldots, r_{n + 1})\) in \(Q\) such that
\begin{enumerate}
    \item \(r_0 = q_0\),
    \item \(r_{i + 1} \in \delta(r_i, w_i)\) for \(i \in \Nln[n + 1]\), and
    \item \(r_{n + 1} \in F\).
\end{enumerate}

\Bth
    \label{thm1}
    Every nondeterministic finite-state automaton has an equivalent
    deterministic finite-state automaton.
\Eth
\Bpr
    Let \(\Sigma\) be an alphabet, let \(A\) be a language over \(\Sigma\), and
    let \(N = (\Sigma, Q, \delta, q_0, F)\) be a nondeterministic finite-state
    automaton recognising \(A\). We construct a deterministic finite-state
    automaton \(M = (\Sigma, Q', \delta', q_0', F')\) which also recognises
    \(A\).

    We first see that \(Q' = \pow(Q)\) and that \(F' = \set{R \in Q' : R \cap F
    \neq \nset}\).

    Let \(\map{\delta_0}{Q \times \set{\epsilon}}{\pow(Q)}\) be defined as
    \(\delta_0(q, \epsilon) = \delta(q, \epsilon)\) for each \(q \in Q\). Assume
    first that, thus induced, \(\delta_0 = \nset\) for \(N\). For each \(R \in
    Q'\) and each \(a \in \Sigma\), let \(\delta'(R, a) = \set*{q \in Q :
    \exists \, r \in R \, \big(q \in \delta(r, a)\big)}\). Equivalently,
    \[
        \delta'(R, a) = \bigcup_{r \in R} \delta(r, a).
    \]
    Also let \(q_0' = \set{q_0'}\). We then see that \(M = (\Sigma, Q', \delta',
    q_0', F')\) recognises \(A\).

    Assume then that \(\delta_0 \neq \nset\) for \(N\). For each \(R \subseteq Q
    \), let
    \[
        E(R) = \set*{q \in Q : \exists \, n \in \setnat \, \exists \, r \in R \,
        \big(q = \delta^n(r, \epsilon)\big)}.
    \]
    We then let
    \[
        \delta'(R, a) = \set*{q \in Q : \exists \, r \in R \, s \in
        E\big(\delta(r, a)\big)}
    \]
    and let \(q_0' = E(\set{q_0})\). We similarly see that \(M = (\Sigma, Q',
    \delta', q_0', F')\) recognises \(A\).

    Therefore, the theorem holds.
\Epr

\subsubsection{Regular Expressions and Regular Languages}

\Bdf
    Let \(\Sigma\) be an alphabet. Then \(R\) is a \emph{regular expression over
    \(\Sigma\)} if
    \begin{enumerate}
        \item \(R = \nset\),
        \item \(R = \epsilon\),
        \item \(R = a\) for some \(a \in \Sigma\),
        \item \(R = R_1 \cup R_2\) wherein \(R_1\) and \(R_2\) are regular
        expressions over \(\Sigma\),
        \item \(R = R_1 R_2\)\footnote{\(R_1 R_2\) denotes the concatenation of
        \(R_1\) and \(R_2\).} wherein \(R_1\) and \(R_2\) are regular
        expressions over \(\Sigma\), or
        \item \(R = R_1^*\) wherein \(R_1\) is a regular expression over
        \(\Sigma\).
    \end{enumerate}
\Edf

The language described by a regular expression is a \emph{regular language},
which is unique. If \(R\) is a regular expression, we denote the regular
language it describes by \(L(R)\).

Let \(\Sigma\) be an alphabet, let \(a \in \Sigma\), and let \(R\), \(R_1\), and
\(R_2\) be regular expressions over \(\Sigma\). If \(R = \nset\), then \(L(R) =
\nset\). If \(R = \epsilon\), then \(L(R) = \set{\epsilon}\). If \(R = a\), then
\(L(R) = \set{a}\). If \(R = R_1 \cup R_2\), then \(L(R) = L(R_1) \cup L(R_2)\).
If \(R = R_1 R_2\), then \(L(R) = L(R_1) L(R_2)\)\footnote{If \(A\) and \(B\)
are languages, \(A B\) denotes the concatenation of \(A\) and \(B\), defined
as \(A B = \set{a b : a \in A \land b \in B}\).}. If \(R = R_1^*\), then \(L(R) =
L(R_1)^*\).

\subsubsection{Equivalence Between Finite-State Automata and Regular Languages}

\Blm
    \label{lem1}
    If a language is regular, then some nondeterministic finite-state automaton
    recognises it.
\Elm
\Bpr
    Let \(\Sigma\) be an alphabet and let \(R\) be a regular expression over
    \(\Sigma\).

    If \(R = \nset\), then the nondeterministic finite-state automaton \(N\)
    characterised by the following diagram recognises \(L(R)\).
    \begin{figure}[!ht]
        \centering
        \begin{tikzpicture}[
            ->,>=latex,auto,semithick,
            node distance=2.5cm
        ]
        \node[initial,state] (q) {\(q\)};
        \end{tikzpicture}
    \end{figure}

    \noindent Equivalently, \(N = (\Sigma, \set{q}, \delta, q, \nset)\) wherein
    \(\delta(r, b) = \nset\) for any \(r\) and \(b\).

    If \(R = \epsilon\), then the nondeterministic finite-state automaton \(N\)
    characterised by the following diagram recognises \(L(R)\).
    \begin{figure}[!ht]
        \centering
        \begin{tikzpicture}[
            ->,>=latex,auto,semithick,
            node distance=2.5cm
        ]
        \node[initial,state,accepting] (q) {\(q\)};
        \end{tikzpicture}
    \end{figure}

    \noindent Equivalently, \(N = (\Sigma, \set{q}, \delta, q, \set{q})\)
    wherein \(\delta(r, b) = \nset\) for any \(r\) and \(b\).

    If \(R = a\) for some \(a \in \Sigma\), then the nondeterministic
    finite-state automaton \(N\) characterised by the following diagram
    recognises \(L(R)\).
    \begin{figure}[!ht]
        \centering
        \begin{tikzpicture}[
            ->,>=latex,auto,semithick,
            node distance=2.5cm
        ]
        \node[initial,state] (q0) {\(q_0\)};
        \node[state,accepting] (q1) [right of=q0] {\(q_1\)};
        \path (q0) edge node {\(a\)} (q1);
        \end{tikzpicture}
    \end{figure}

    \noindent Equivalently, \(N = (\Sigma, \set{q_0, q_1}, \delta, q_0,
    \set{q_1})\) wherein \(\delta(q_0, a) = \set{q_1}\) and \(\delta(r, b) =
    \nset\) if \(r \neq q_0\) or \(b \neq a\).

    Assume that \(R_1\) and \(R_2\) are regular expressions over \(\Sigma\),
    that \(N_1 = (\Sigma, Q_1, \delta_1, q_1, F_1)\) is a nondeterministic
    finite-state automaton recognising \(L(R_1)\), and that \(N_2 = (\Sigma,
    Q_2, \delta_2, q_2, F_2)\) is a nondeterministic finite-state automaton
    recognising \(L(R_2)\).

    If \(R = R_1 \cup R_2\), let \(\set{q_0}\) be disjoint from \(Q_1\) and
    \(Q_2\), let \(Q = Q_1 \cup Q_2 \cup \set{q_0}\), and let \(F = F_1 \cup
    F_2\). Define \(\map{\delta}{Q \times (\Sigma \cup
    \set{\epsilon})}{\pow(Q)}\) so that for each \(r \in Q\) and each \(b \in
    \Sigma \cup \set{\epsilon}\) we have
    \[
        \delta(r, b) = \begin{cases}
            \delta_1(r, b) & \text{if }\ r \in Q_1,\\
            \delta_2(r, b) & \text{if }\ r \in Q_2,\\
            \set{q_1, q_2} & \text{if }\ r = q_0 \land b = \epsilon \text{,
            and}\\
            \nset & \text{otherwise.}
        \end{cases}
    \]
    We see that \(N = (\Sigma, Q, \delta, q_0, F)\) is a nondeterministic
    finite-state automaton recognising \(L(R)\).

    If \(R = R_1 R_2\), let \(Q = Q_1 \cup Q_2\). Define \(\map{\delta}{Q \times
    (\Sigma \cup \set{\epsilon})}{\pow(Q)}\) so that for each \(r \in Q\) and
    each \(b \in \Sigma \cup \set{\epsilon}\) we have
    \[
        \delta(r, b) = \begin{cases}
            \delta_1(r, b) & \text{if }\ (r \in Q_1 \land r \not\in F_1) \lor (r
            \in F_1 \land b \neq \epsilon),\\
            \delta_1(r, b) \cup \set{q_2} & \text{if }\ r \in F_1 \land b =
            \epsilon \text{, and}\\
            \delta_2(r, b) & \text{otherwise.}
        \end{cases}
    \]
    We see that \(N = (\Sigma, Q, \delta, q_1, F_2)\) is a nondeterministic
    finite-state automaton recognising \(L(R)\).

    If \(R = R_1^*\), let \(\set{q_0}\) be disjoint from \(Q_1\), let \(Q = Q_1
    \cup \set{q_0}\), and let \(F = F_1 \cup \set{q_0}\). Define
    \(\map{\delta}{Q \times (\Sigma \cup \set{\epsilon})}{\pow(Q)}\) so that for
    each \(r \in Q\) and each \(b \in \Sigma \cup \set{\epsilon}\) we have
    \[
        \delta(r, b) = \begin{cases}
            \delta_1(r, b) & \text{if }\ r \in Q_1 \setminus F_1 \lor (r \in F_1
            \land b \neq \epsilon),\\
            \delta_1(r, b) \cup \set{q_1} & \text{if }\ r \in F_1 \land b =
            \epsilon,\\
            \set{q_1} & \text{if }\ r = q_0 \land b = \epsilon \text{, and}\\
            \nset & \text{otherwise.}
        \end{cases}
    \]
    We see that \(N = (\Sigma, Q, \delta, q_0, F)\) is a nondeterministic
    finite-state automaton recognising \(L(R)\).

    Therefore, the lemma holds by the principle of induction.
\Epr

\Bdf
    A \emph{generalised nondeterministic finite-state automaton} is an ordered
    quintuple \((\Sigma, Q, \delta, q_0, q_1)\) wherein
    \begin{enumerate}
        \item \(\Sigma\) is an alphabet,
        \item \(Q\) is a finite set of states,
        \item \(\map{\delta}{(Q \setminus \set{q_1}) \times (Q \setminus
        \set{q_0})}{\mathcal{R}}\) wherein \(\mathcal{R}\) is the set of all
        regular expressions over \(\Sigma\) is the transition function,
        \item \(q_0 \in Q\) is the initial state, and
        \item \(q_1 \neq q_0 \in Q\) is the accepting state.
    \end{enumerate}
\Edf

Let \(G = (\Sigma, Q, \delta, q_0, q_1)\) be a generalised nondeterministic
finite-state automaton and let \(w\) be a word over \(\Sigma\). Then \(M\)
accepts \(w\) if \(w = w_0 \cdots w_n\) wherein \(n \in \setnat\) such that each
\(w_i \in \Sigma^*\) for some \(i \in \Nln[n + 1]\) and that there exists a
sequence of states \((r_0, \ldots, r_{n + 1})\) in \(Q\) such that
\begin{enumerate}
    \item \(r_0 = q_0\),
    \item \(r_{n + 1} = q_1\), and
    \item \(w_i \in L\big(\delta(r_i, r_{i + 1})\big)\) for \(i \in \Nln[n +
    1]\).
\end{enumerate}

\Blm
    \label{lem2}
    If a nondeterministic finite-state automaton recognises a language, then it
    is regular.
\Elm
\Bpr
    Let \(\Sigma\) be an alphabet, let \(A\) be a language over \(\Sigma\), and
    let \(N = (\Sigma, Q, \delta, q_0, F)\) be a nondeterministic finite-state
    automaton recognising \(A\). We argue that \(A\) is described by some
    regular expression \(R\) over \(\Sigma\).

    Let \(G = (\Sigma, Q', \delta', q_0', q_1')\) be a generalised
    nondeterministic finite-state automaton such that
    \begin{enumerate}
        \item \(\set{q_0', q_0'} \cap Q = \nset\),
        \item \(Q' = Q \cup \set{q_0', q_1'}\), and
        \item for each \(r_0 \in Q' \setminus \set{q_1'}\) and each \(r_1 \in Q'
        \setminus \set{q_0'}\) we have
            \[
                \delta'(r_0, r_1) = \begin{cases}
                        \epsilon & \text{if }\ (r_0 = q_0' \land r_1 = q_0) \lor
                        (r_0 \in F \land r_1 = q_1'),\\
                        R' & \text{if }\ r_0 \in Q \land r_1 \in Q \land \forall
                        \, r \in L(R') \, \big(r_1 \in \delta(r_0, r)\big)
                        \text{, and}\\
                        \nset & \text{otherwise.}
                    \end{cases}
            \]
    \end{enumerate}
    We see that \(G\) also recognises \(A\). We shall then convert \(G\) into
    regular expression \(R\).

    Let \(k = \abs{Q'}\).

    If \(k = 2\), then \(Q' = \set{q_0', q_1'}\), and so \(R = \delta'(q_0',
    q_1')\) is the regular expression.

    If \(k > 2\), let \(q \in Q'\) be distinct from \(q_0'\) and \(q_1'\), and
    let \(G' = (\Sigma, Q'', \delta'', q_0', q_1')\) be a generalised
    nondeterministic finite-state automaton such that
    \begin{enumerate}
        \item \(Q'' = Q' \setminus \set{q}\),
        \item for each \(r_0 \in Q'' \setminus \set{q_0'}\) and each \(r_1 \in
        Q'' \setminus \set{q_1'}\) we have
        \[
            \delta''(r_0, r_1) = R_0 R_1^* R_2 \cup R_3
        \]
        wherein \(R_0 = \delta'(r_0, q)\), \(R_1 = \delta'(q, q)\), \(R_2 =
        \delta'(q, r_1)\), and \(R_3 = \delta'(r_0, r_1)\).
    \end{enumerate}
    We see that \(G'\) is equivalent to \(G\).

    Because \(G'\) has one fewer state than \(G\), by the principle of
    induction, there exists regular expression \(R\) converted from \(G\) for
    any generalised nondeterministic finite-state automaton.

    Therefore, the lemma holds.
\Epr

\Bth
    \label{thm2}
    A language is regular if and only if some nondeterministic finite-state
    automaton recognises it.
\Eth
\Bpr
    The theorem holds by \autoref{lem1} and \autoref{lem2}.
\Epr

\Bcr
    A language is regular if and only if some deterministic finite-state
    automaton recognises it.
\Ecr
\Bpr
    The corollary holds by \autoref{thm1} and \autoref{thm2}.
\Epr

\subsubsection{Nonregular Languages}

\Bth[pumping lemma]
    Let \(\Sigma\) be an alphabet. If \(A\) is a regular language over
    \(\Sigma\), then there is some \(p \in \posint\), the
    \textbf{\textit{pumping length}}, such that if \(w \in A\) satisfies
    \(\abs{w} \geq p\), then there exist \(x\), \(y\), and \(z \in \Sigma^*\)
    which satisfy
    \begin{enumerate}
        \item \(w = x y z\),
        \item \(x y^i z \in A\) for each \(i \in \setnat\),
        \item \(\abs{y} > 0\), and
        \item \(\abs{x y } \leq p\).
    \end{enumerate}
\Eth
\Bpr
    Let \(M = (\Sigma, Q, \delta, q_0, F)\) be a deterministic finite-state
    automaton recognising \(A\) and let \(p = \abs{Q}\).

    Let \(w = w_0 \cdots w_n\) wherein \(n \in \setnat\) be a word in \(R\) of
    length \(n + 1\) which satisfies \(n + 1 \geq p\). Let \((r_0, \ldots, r_{n
    + 1})\) be the sequence of states that \(M\) enters when accepting \(w\).
    This sequence has length \(n + 2\), which must be at least \(p + 1\). Among
    the first \(p + 1\) elements in the sequence, two must be the same state by
    the pigeonhole principle. Let the first of these be \(r_i\) and the second
    \(r_j\). We note that \(i \leq j - 1\) and that \(j \leq p\). Now let \(x =
    w_0 \cdots w_{i - 1}\), \(y = w_i \cdots w_{j - 1}\), and \(z = w_j \cdots
    w_n\).

    Thus induced, \(w = x y z\) satisfies the pumping lemma.
\Epr

\Bxr
    Let \(\Sigma = \set{0, 1}\) be an alphabet. Prove that the language
    \(A = \set{0^n 1^n : n \in \setnat}\) is not regular.
\Exr
\Bsl
    Assume for the sake of contradiction that \(A\) is regular. Let \(p\) be the
    pumping length thereof, and let \(w = 0^p 1^p\). Then there exist \(x\),
    \(y\), and \(z \in \Sigma^*\) such that \(w = x y z\), that \(x y^i z \in
    A\) for \(i \in \setnat\), that \(\abs{y} > 0\), and that \(\abs{x y} \leq
    p\) by the pumping lemma. We argue that it is impossible that there exist
    such words.

    We first see that \(y = 0^j\) wherein \(j \in \posint\), for \(\abs{y} > 0\)
    and \(\abs{x y} \leq p\). Thus, \(x y y z = 0^{p + j} 1^p \not\in A\), which
    is a contradiction of \(x y^i z \in A\) for \(i \in \setnat\).

    By the contradiction obtained above, the original proposition holds.
\Esl

\subsection{Pushdown Automata and Context-Free Languages}

\subsubsection{Pushdown Automata}

\Bdf
    A \emph{pushdown automaton} is an ordered sextuple \((\Sigma, \Gamma, Q,
    \delta, q_0, F)\) wherein
    \begin{enumerate}
        \item \(\Sigma\) is an alphabet for the input,
        \item \(\Gamma\) is another alphabet for the \emph{stack},
        \item \(Q\) is a finite set of states,
        \item \(\map{\delta}{Q \times (\Sigma \cup \set{\epsilon}) \times
        (\Gamma \cup \set{\epsilon})}{\pow\big(Q \times (\Gamma \cup
        \set{\epsilon})\big)}\) is the transition function,
        \item \(q_0 \in Q\) is the initial state, and
        \item \(F \subseteq Q\) is the set of accepting states.
    \end{enumerate}
\Edf

Let \(P = (\Sigma, \Gamma, Q, \delta, q_0, F)\) be a pushdown automaton and let
\(w\) be a word over \(\Sigma\). Then \(M\) accepts \(w = w_0 \cdots w_n\)
wherein \(n \in \setnat\) such that \(w_i \in \Sigma \cup \set{\epsilon}\) for
some \(i \in \Nln[n + 1]\) and that there exist a sequence of states \((r_0,
\ldots, r_{n + 1})\) in \(Q\) and a sequence of words \((s_0, \ldots, s_{n +
1})\) in \(\Gamma^*\) such that
\begin{enumerate}
    \item \(r_0 = q_0\),
    \item \(s_0 = \epsilon\),
    \item for each \(i \in \Nln[n + 1]\) there exist some \(a\) and \(b \in
    \Gamma \cup \set{\epsilon}\) and some \(t \in \Gamma^*\) such that \((r_{i +
    1}, b) \in \delta(r_i, w_i, a)\), that \(s_i = a t\), and that\(s_{i + 1} =
    b t\) , and \item \(r_{n + 1} \in F\).
\end{enumerate}

\Bxr
    Let \(\Sigma = \set{0, 1}\) be an alphabet. Construct a pushdown automaton
    which recognises the language \(A = \set{0^n 1^n : n \in \setnat}\).
\Exr
\Bsl
    The pushdown automaton \(P\) characterised by the following diagram
    recognises \(A\).
    \begin{figure}[!ht]
        \centering
        \begin{tikzpicture}[
            ->,>=latex,auto,semithick,
            node distance=2.5cm
        ]
        \node[initial,state,accepting] (q0) {\(q_0\)};
        \node[state] (q1) [right of=q0] {\(q_1\)};
        \node[state] (q2) [below of=q1] {\(q_2\)};
        \node[state,accepting] (q3) [left of=q2] {\(q_3\)};
        \path (q0) edge node {\(\epsilon, \epsilon \to \$\)} (q1)
        (q1) edge node {\(1, 0 \to \epsilon\)} (q2)
        edge [loop right] node {\(0, \epsilon \to 0\)} ()
        (q2) edge node {\(\epsilon, \$ \to \epsilon\)} (q3)
        edge [loop right] node {\(1, 0 \to \epsilon\)} ();
        \end{tikzpicture}
    \end{figure}

    \noindent Equivalently, \(P = (\Sigma, \Gamma, Q, \delta, q_0, F)\)
    wherein
    \begin{enumerate}
        \item \(\Gamma = \set{0, \$}\),
        \item \(Q = \set{q_0, q_1, q_2, q_3}\),
        \item \(F = \set{q_0, q_3}\), and
        \item for each \(q \in Q\), each \(b \in \Sigma \cup
            \set{\epsilon}\), and each \(s \in \Gamma \cup
            \set{\epsilon}\) we have
            \[
                \delta(q, b, s) = \begin{cases}
                    \set{(q_1, \$)} & \text{if } q = q_0 \land b = \epsilon
                    \land s = \epsilon,\\
                    \set{(q_1, 0)} & \text{if } q = q_1 \land b = 0 \land s =
                    \epsilon,\\
                    \set{(q_2, \epsilon)} & \text{if } (q = q_1 \lor q = q_2)
                    \land b = 1 \land s = 0,\\
                    \set{(q_3, \epsilon)} & \text{if } q = q_2 \land b =
                    \epsilon \land s = \$ \text{, and}\\
                    \nset & \text{otherwise}
                \end{cases}
            \]
    \end{enumerate}
    is a pushdown automaton which recognises \(A\).
\Esl

\subsubsection{Context-Free Grammars and Context-Free Languagse}

\Bdf
    A \emph{context-free grammar} is an ordered quadruple \((\Sigma, V, R, S)\)
    wherein
    \begin{enumerate}
        \item \(\Sigma\) is an alphabet of \emph{terminals},
        \item \(V\) is another alphabet of \emph{variables}, which is disjoint
        from \(\Sigma\),
        \item \(\map{R}{V}{(\Sigma \cup V)^*}\) is a finite set of \emph{production rules}, and
        \item \(S \in V\) is the \emph{start variable}.
    \end{enumerate}
\Edf

Let \((\Sigma, V, R, S)\) be a context-free grammar. If \(R(A) = w\) wherein \(A
\in V\) and \(w \in (\Sigma \cup V)^*\) is a production rule, we write \(A \to
w\). Let \(u\), \(v\), and \(w \in (\Sigma \cup V)^*\). If \(A \to w\) is a
production rule, we say that \(u A v\) \emph{yields} \(u w v\) and write \(u A v
\Rightarrow u w v\). We say that \(u\) \emph{derives} \(v\) and write \(u
\Rightarrow^* v\) if \(u = v\), \(u \Rightarrow v\), or there exists a sequence
\((u_0, \ldots, u_n)\) in \((\Sigma \cup V)^*\) for some \(n \in \setnat\) such
that
\[
    u \Rightarrow u_0 \Rightarrow \cdots \Rightarrow u_n \Rightarrow v.
\]
If \(A \to u\) and \(A \to v\) are production rules of the grammar, we may
denote them by \(A \to u \, | \, v\). The \emph{language generated by the
grammar} is \(\set{w \in \Sigma^* : S \Rightarrow^* w}\).

The language generated by a context-free grammar is a \emph{context-free
language}.

\Bxr
    Let \(\Sigma = \set{0, 1}\) be an alphabet. Construct a context-free grammar
    which generates the language \(A = \set{0^n 1^n : n \in \setnat}\).
\Exr
\Bsl
    Let \((\Sigma, V, R, S)\) be the context-free grammar wherein \(V =
    \set{S}\) and \(R\) consists of the following production rule
    \[
        S \to 0 S 1 \, | \, \epsilon.
    \]
    The language generated by the above context-free grammar is \(A\).
\Esl

A derivation of a word in a context-free grammar is a \emph{leftmost derivation}
if at every step of production the leftmost remaining variable is the one
substituted according to a production rule.

\Bdf
    A word is derived \emph{ambiguously} in a context-free grammar if there exist
    two or more distinct leftmost derivations for it.

    A context-free grammar is \emph{ambiguous} is it generates some words
    ambiguously.
\Edf

Some context-free languages can only be generated by ambiguous context-free
grammars. Such languages are \emph{inherently ambiguous}.

\subsubsection{Chomsky Normal Form}

\Bdf
    A context-free grammar is \emph{in Chomsky normal form} if every production
    rule thereof is
    \begin{enumerate}
        \item \(S \to \epsilon\) wherein \(S\) is the start variable,
        \item \(A \to B C\) wherein \(A\), \(B\), and \(C\) are variables and
        \(B\) and \(C\) are not the start variable, or
        \item \(A \to a\) wherein \(A\) is a variable and \(a\) is a terminal.
    \end{enumerate}
\Edf

\Bth
    Any context-free language is generated by a context-free grammar in Chomsky
    normal form.
\Eth
\Bpr
    Let \((\Sigma, V, R, S)\) be a context-free grammar. We demonstrate a
    procedure to convert it into another context-free grammar in Chomsky normal
    form \((\Sigma, V', R', S')\).

    We first add \(S' \to S\) as a production rule.

    Second, if there exist rules of the form \(A \to \epsilon\) wherein \(A \neq
    S'\), we remove them and repeatedly replace any rule of the form \(B \to u A
    v\) wherein \(B \in V'\) and \(u\) and \(v \in (\Sigma \cup V')^*\) with \(B
    \to u v\) for each occurrence of \(A\).

    Third, if there exist rules of the form \(A \to B\) wherein \(A\) and \(B
    \in V'\), we remove them and replace any rule of the form \(B \to u\)
    wherein \(u \in (\Sigma \cup V')^*\) with \(A \to u\).

    Lastly, we replace each rule of the form \(A \to u_0 \cdots u_n\) wherein
    \(n \in \setnat\) and \(u_i \in \Sigma \cup V'\) for \(i \in \Nln[n + 1]\)
    such that \(n > 1\) with the rules \(A \to u_0 A_0\), \(A_0 \to u_1 A_1\),
    \ldots, \(A_{n - 2} \to u_{n - 1} u_n\) and add \(A_i\) for \(i \in \Nln[n -
    1]\) as variables. We then replace any terminal \(u_i\) for \(i \in \Nln[n +
    1]\) with the new variable \(U_i\) while adding the rule \(U_i \to u_i\).

    The resultant context-free grammar is in Chomsky normal form, and thus the
    theorem holds.
\Epr

\subsubsection{Equivalence Between Pushdown Automata and Context-Free Languages}

\Blm
    \label{lem3}
    If a language is context-free, then some pushdown automaton recognises it.
\Elm
\Bpr
    Let \(\Sigma\) be an alphabet, let \(A\) be a context-free language over
    \(\Sigma\), and let \(G = (\Sigma, V, R, S)\) be a context-free grammar
    which generates \(A\).  We construct a pushdown automaton \(P = (\Sigma,
    \Gamma, Q, \delta, q_0, F)\) which recognises \(A\).

    Let \(b \in \Sigma \cup \set{\epsilon}\), let \(s \in \Gamma \cup
    \set{\epsilon}\), and let \(q\) and \(r \in Q\). Let \(u = u_0 \cdots u_i\)
    wherein \(i \in \setnat\) be a word over \(\Gamma\). We denote by \((r, u)
    \in \delta(q, b, s)\) that there exist a sequence \((q_0, \ldots, q_{i -
    1})\) in \(Q\) such that
    \begin{enumerate}
        \item \((q_0, u_i) \in \delta(q, b, s)\),
        \item \(\set{(q_{j + 1}, u_{i - j - 1})} = \delta(q_j, \epsilon,
        \epsilon)\) for \(j \in \Nln[i - 1]\), and
        \item \(\set{(r, u_0)} = \delta(q_{i - 1}, \epsilon, \epsilon)\).
    \end{enumerate}

    Let \(Q = E \cup \set{q_0, q_1, q_2}\) and let \(F = \set{q_2}\). Let
    \(\set{\$}\) be disjoint from \(\Sigma\) and \(V\), and let \(\Gamma =
    \Sigma \cup V \cup \set{\$}\). Let \(\delta\) be defined as
    \[
        \delta(q, b, s) = \begin{cases}
            \set{(q_1, S \$)} & \text{if } q = q_0 \land b = \epsilon \land s =
            \epsilon,\\
            \set{(q_1, w)} & \text{if } q = q_1 \land b = \epsilon \land s = A
            \land (A \to w) \in R,\\
            \set{(q_1, \epsilon)} & \text{if } q = q_1 \land b = a \land s = a
            \in \Sigma,\\
            \set{(q_2, \epsilon)} & \text{if } q = q_1 \land b = \epsilon \land
            s = \$\text{, and},\\
            \nset & \text{otherwise.}
        \end{cases}
    \]
    Let \(E \subseteq Q\) consist of those states necessary to make the
    \(\delta\) as described above well-defined per the notation given in the
    previous paragraph.

    The following diagram illustrates the constructed \(P\).
    \begin{figure}[!ht]
        \centering
        \begin{tikzpicture}[
            ->,>=latex,auto,semithick,
            node distance=2.5cm
        ]
        \node[initial,state] (q0) {\(q_0\)};
        \node[state] (q1) [below of=q0] {\(q_1\)};
        \node[state,accepting] (q2) [below of=q1] {\(q_2\)};
        \path (q0) edge node {\(\epsilon, \epsilon \to S \$\)} (q1)
        (q1) edge [loop right] node {\begin{tabular}{l l}
            \(\epsilon, A \to w\) & for production rule \(A \to w\)\\
            \(a, a \to \epsilon\) & for terminal \(a\)
        \end{tabular}} ()
        edge node {\(\epsilon, \$ \to \epsilon\)} (q2);
        \end{tikzpicture}
    \end{figure}

    Thus defined, the pushdown automaton \(P\) recognises \(A\). Therefore, the
    lemma holds.
\Epr

\Blm
    \label{lem4}
    If a pushdown automaton recognises a language, then it is context-free.
\Elm
\Bpr
    Let \(P = (\Sigma, \Gamma, Q, \delta, q_0, F)\) be a pushdown automaton. We
    construct a context-free grammar \(G = (\Sigma, V, R, S)\) which generates
    all words over \(\Sigma\) accepted by \(P\).

    We first let \(P' = (\Sigma, \Gamma, Q, \delta', q_0, F')\) be a pushdown
    automaton equivalent to \(P\) such that
    \begin{enumerate}
        \item \(F' = \set{q_1}\),
        \item there exist \(q \in Q\), \(b \in \Sigma \cup \set{\epsilon}\), and
        \(s \in \Gamma \cup \set{\epsilon}\) which satisfy \(\set{q_1, \epsilon}
        \in \delta'(q, b, s)\), and
        \item if \(\set{r_1, s_1} \in \delta(r_0, b, s_0)\) for some \(r_0\) and
        \(r_1 \in Q\), some \(b \in \Sigma \cup \set{\epsilon}\), and some
        \(s_0\) and \(s_1 \in \Gamma \cup \set{\epsilon}\), then \(s_0 =
        \epsilon\) or \(s_1 = \epsilon\).
    \end{enumerate}

    TODO
\Epr

\Bth
    A language is context-free if and only if some pushdown automaton recognises
    it.
\Eth
\Bpr
    The theorem holds by \autoref{lem3} and \autoref{lem4}.
\Epr

\Bcr
    Every regular language is context-free.
\Ecr
\Bpr
    Let \(\Sigma\) be an alphabet and let \(A\) be a regular language over
    \(\Sigma\). Let \((\Sigma, Q, \delta, q_0, F)\) be a nondeterministic
    finite-state automaton recognising \(A\). Then the pushdown automaton
    \((\Sigma, \nset, Q, \delta', q_0, F)\) wherein \(\delta'(q, b, \epsilon) =
    \delta(q, b)\) for each \(q \in Q\) and each \(b \in \Sigma \cup
    \set{\epsilon}\) also recognises \(A\). Thus, \(A\) is context-free.
\Epr

\Bxr
    Let \(\Sigma = \set{0, 1, 2, 3, 4, 5, 6, 7, 8, 9}\) be an alphabet. Then \(R
    = (1 \cup 2 \cup 3 \cup 4 \cup 5 \cup 6 \cup 7 \cup 8 \cup 9) (0 \cup 1 \cup
    2 \cup 3 \cup 4 \cup 5 \cup 6 \cup 7 \cup 8 \cup 9)^*\) is a regular
    expression over \(\Sigma\), and \(L(R)\) is the set of positive integers in
    base \(10\) written in the Indo--Arabic numeral system.

    Construct a context-free grammar which generates \(L(R)\).
\Exr
\Bsl
    The context-free grammar \((\Sigma, V, R, S)\) wherein \(V = \set{S, A, B}\)
    and \(R\) consists of the production rules
    \begin{align*}
        S & \to A B^*\\
        A & \to 1 \, | \, 2 \, | \, 3 \, | \, 4 \, | \, 5 \, | \, 6 \, | \, 7 \,
        | \, 8 \, | \, 9\\
        B & \to A \, | \, 0\\
    \end{align*}
    generates \(L(R)\).
\Esl

\subsubsection{Non-Context-Free Languages}

\Bth[pumping lemma for context-free languages]
    Let \(\Sigma\) be an alphabet. If \(A\) is a context-free language over
    \(\Sigma\), then there is some \(p \in \posint\), the pumping length, such
    that if \(w \in A\) satisfies \(\abs{w} \geq p\), then there exist \(u\),
    \(v\), \(x\), \(y\), and \(z \in \Sigma^*\) which satisfy
    \begin{enumerate}
        \item \(w = u v x y z\),
        \item \(u v^i x y^i z \in A\) for each \(i \in \setnat\),
        \item \(\abs{v y} > 0\), and
        \item \(\abs{v x y} \leq p\).
    \end{enumerate}
\Eth
\Bpr
    TODO
\Epr

\subsubsection{Deterministic Pushdown Automata and Deterministic Context-Free
Languages}

\Edc
