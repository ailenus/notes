\documentclass[12pt]{article}

\makeatletter
\def\@maketitle {
    \newpage
    \null
    \vskip 2em
    \begin{center}
        \let \footnote \thanks
        {
            \Large\bfseries \MakeUppercase{\@title} \par
        }
        \vskip 1.5em
        {
            \large \lineskip .5em \begin{tabular}[t]{c}
                \MakeUppercase{\@author}
            \end{tabular} \par
        }
        \vskip 1.5em
        {
            \large\scshape \@date
        }
    \end{center}
    \par
    \vskip 1.5em
}
\makeatother

%%%%%%%%%%%%%%%%%%%%%%%%%%%%%%%%%%%%%%%%%%%%%%%%%%%%%%%%%%%%%%%%%%%%%%%%%%%%%%%
%% Packages
%%%%%%%%%%%%%%%%%%%%%%%%%%%%%%%%%%%%%%%%%%%%%%%%%%%%%%%%%%%%%%%%%%%%%%%%%%%%%%%

%% basic
\usepackage[a4paper,margin=1in]{geometry}
\usepackage{fontspec}
\usepackage{polyglossia}

%% formatting
\usepackage{setspace}
\usepackage{enumitem}
\usepackage{tocloft}
\usepackage{titlesec}
\usepackage{fancyhdr}

%% maths
\usepackage{amsmath}
\usepackage{amsthm}
\usepackage{mathtools}
\usepackage[math-style=ISO,bold-style=ISO]{unicode-math}

%% utilities
\usepackage{array}
\usepackage{booktabs}
\usepackage[colorlinks=true]{hyperref}

%%%%%%%%%%%%%%%%%%%%%%%%%%%%%%%%%%%%%%%%%%%%%%%%%%%%%%%%%%%%%%%%%%%%%%%%%%%%%%%
%% Settings
%%%%%%%%%%%%%%%%%%%%%%%%%%%%%%%%%%%%%%%%%%%%%%%%%%%%%%%%%%%%%%%%%%%%%%%%%%%%%%%

%% basic
\setmainlanguage[variant=british]{english}
\SetLanguageKeys{english}{indentfirst=true}
\setmainfont{STIX Two Text}
\setmathfont{STIX Two Math}

%% formatting
\doublespacing
\setlist{noitemsep}
\setlist[1]{labelindent=\parindent}
\setlist[1]{listparindent=\parindent}
\setlist[enumerate,1]{label=(\alph*)}
\setlist[enumerate,2]{label=(\roman*)}
\setlength{\headheight}{15.2pt}
\setlength{\cftsecindent}{0em}
\setlength{\cftsubsecindent}{0em}
\setlength{\cftsubsubsecindent}{0em}
\renewcommand*{\cfttoctitlefont}{\large\scshape}
\renewcommand*{\cftsecfont}{\normalfont}
\renewcommand*{\cftsecpagefont}{\normalfont}
\renewcommand*{\cftsecleader}{\cftdotfill{\cftdotsep}}
\titleformat{\section}{
    \normalfont\Large\scshape\filcenter
}{}{1em}{}
\titleformat{\subsection}{
    \normalfont\large\scshape\raggedright
}{\thesubsection}{1em}{}
\titleformat{\subsubsection}{
    \normalfont\normalsize\scshape\raggedright
}{\thesubsubsection}{1em}{}
\setcounter{footnote}{1}

%% maths
\makeatletter
\newtheoremstyle{dfn}
    {\topsep}
    {\topsep}
    {\upshape\addtolength{\@totalleftmargin}{3em}
    \addtolength{\linewidth}{-6em}
    \parshape 1 3em \linewidth}
    {0pt}
    {\scshape}
    {.}
    {5pt plus 1pt minus 1pt}
    {}
\newtheoremstyle{thm}
    {\topsep}
    {\topsep}
    {\slshape\addtolength{\@totalleftmargin}{3em}
    \addtolength{\linewidth}{-6em}
    \parshape 1 3em \linewidth}
    {0pt}
    {\bfseries}
    {.}
    {5pt plus 1pt minus 1pt}
    {}
\newtheoremstyle{xrc}
    {\topsep}
    {\topsep}
    {\upshape\addtolength{\@totalleftmargin}{3em}
    \addtolength{\linewidth}{-6em}
    \parshape 1 3em \linewidth}
    {0pt}
    {\bfseries\scshape}
    {.}
    {5pt plus 1pt minus 1pt}
    {}
\newtheoremstyle{sol}
    {\topsep}
    {\topsep}
    {\upshape}
    {0pt}
    {\itshape}
    {.}
    {5pt plus 1pt minus 1pt}
    {}
\makeatother
\theoremstyle{dfn}
\newtheorem{dfn}{Definition}
\theoremstyle{thm}
\newtheorem{thm}{Theorem}
\newtheorem{lem}{Lemma}
\newtheorem{cor}{Corollary}
\renewcommand\qedsymbol{\(\mdlgwhtsquare\)}
\theoremstyle{xrc}
\newtheorem{xrc}{Exercise}
\theoremstyle{sol}
\newtheorem*{XxmpX}{Solution}
\newenvironment{solution}{
    \renewcommand{\qedsymbol}{\(\mdlgwhtlozenge\)}
    \pushQED{\qed}\begin{XxmpX}
}{\popQED\end{XxmpX}}
\def\equationautorefname~#1\null{(#1)\null}
\newcommand{\dfnautorefname}{definition}
\newcommand{\thmautorefname}{theorem}
\newcommand{\lemautorefname}{lemma}
\newcommand{\corautorefname}{corollary}

%% aliases
\newcommand*{\Bdc}[2][\today]{
    \title{#2}
    \author{Yannan Mao}
    \date{#1}
    \lhead{\scshape #2}
    \rhead{\scshape Yannan Mao}
    \begin{document}
    \maketitle
    \tableofcontents
    \thispagestyle{empty}
    \clearpage
    \pagestyle{fancy}
    \setcounter{page}{1}
}
\newcommand*{\Edc}{\end{document}}
\newcommand*{\Bdf}{\begin{dfn}}
\newcommand*{\Edf}{\end{dfn}}
\newcommand*{\Bth}{\begin{thm}}
\newcommand*{\Eth}{\end{thm}}
\newcommand*{\Blm}{\begin{lem}}
\newcommand*{\Elm}{\end{lem}}
\newcommand*{\Bcr}{\begin{cor}}
\newcommand*{\Ecr}{\end{cor}}
\newcommand*{\Bxr}{\begin{xrc}}
\newcommand*{\Exr}{\end{xrc}}
\newcommand*{\Bpr}{\begin{proof}}
\newcommand*{\Epr}{\end{proof}}
\newcommand*{\Bsl}{\begin{solution}}
\newcommand*{\Esl}{\end{solution}}

\DeclareMathOperator{\dom}{dom}             % domain
\DeclareMathOperator{\im}{im}               % image
\DeclareMathOperator*{\argmax}{arg\,max}    % arg max
\DeclareMathOperator*{\argmin}{arg\,min}    % arg min

\DeclarePairedDelimiter{\abs}{\lvert}{\rvert}               % absolute value, cardinality
\DeclarePairedDelimiter{\norm}{\lVert}{\rVert}              % norm
\DeclarePairedDelimiterX{\inn}[2]{\langle}{\rangle}{#1,#2}  % inner product
\DeclarePairedDelimiter{\set}{\{}{\}}                       % set

\newcommand*{\ee}{\ensuremath{\mathrm{e}}}              % e
\newcommand*{\ii}{\ensuremath{\mathrm{i}}}              % i
\newcommand*{\diff}{\ensuremath{\mathop{}\!\mathrm{d}}} % differential

\newcommand*{\pow}{\ensuremath{\mathcal{P}}}            % power set
\newcommand*{\nset}{\ensuremath{\emptyset}}             % null set
\newcommand*{\setnat}{\ensuremath{\mathbb{N}}}          % set of natural numbers
\newcommand*{\setint}{\ensuremath{\mathbb{Z}}}          % set of integers
\newcommand*{\setrat}{\ensuremath{\mathbb{Q}}}          % set of rational numbers
\newcommand*{\setreal}{\ensuremath{\mathbb{R}}}         % set of real numbers
\newcommand*{\setcomp}{\ensuremath{\mathbb{C}}}         % set of complex numbers
\newcommand*{\posint}{\ensuremath{\setint_{>0}}}        % set of positive integers
\newcommand*{\posreal}{\ensuremath{\setreal_{>0}}}      % set of positive real numbers
\newcommand*{\nonneg}{\ensuremath{\setreal_{\ge 0}}}    % set of nonnegative numbers
\newcommand*{\Nln}[1][n]{\ensuremath{\setnat_{<#1}}}    % set of natural numbers less than n

\newcommand*{\vct}[1]{\ensuremath{\symbf{#1}}}      % vector
\newcommand*{\map}[3]{\ensuremath{#1\colon#2\to#3}} % map
\newcommand*{\grad}{\ensuremath{\nabla}}            % gradient

\newcommand*{\dis}{\ensuremath{\displaystyle}}


\Bdc{Notes on Syntax}

\section{Syntactic Categories}

The \emph{syntactic category} of a word, also known as its \emph{part of speech} or \emph{word class}, is determined by
its \emph{distribution}. We note syntactic categories including
\begin{enumerate}
  \item \emph{nouns}, denoted by N;
  \item \emph{verbs}, denoted by V;
  \item \emph{adjectives}, denoted by Adj;
  \item \emph{adverbs}, denoted by Adv;
  \item \emph{prepositions}, denoted by P;
  \item \emph{determiners}, denoted by D;
  \item \emph{conjunctions}, denoted by Conj;
  \item \emph{complementisers}, denoted by C;
  \item \emph{tense markers}, denoted by T; and
  \item \emph{negation}, denoted by Neg.
\end{enumerate}

\subsection{Distribution}

\subsubsection{Morphological Distribution}

The smallest meaningful units in language are referred to as \emph{morphemes}. A \emph{free morpheme} can occur
independently, and a \emph{bound morpheme} or \emph{affix} must attach to another unit, referred to as the \emph{stem}.
An affix attached at the end of the stem is a \emph{suffix}; at the start, a \emph{prefix}; inside, an \emph{infix}; and
at both the start and the end, a \emph{circumfix}.

The stem required by an affix is referred to as its \emph{complement}. An affix requires its complement to be of a
certain syntactic category, a process referred to as \emph{c-selection} or \emph{category selection}. That affixes
c-selects complements of certain syntactic categories form the \emph{morphological distribution} of those categories.

A \emph{lexeme} is a unit of meaning underlying a set of word forms related by inflection. An affix which modifies the
lexeme of its complement is a \emph{derivational morpheme}, and one which retains the lexeme thereof is an
\emph{inflectional morpheme}.

\Bxm
  The English suffix \textit{-ise} is a derivational morpheme and c-selects N or Adj.
\Exm

\subsubsection{Syntactic Distribution}

A syntactic category occurs in certain syntactic structures, referred to as its \emph{syntactic distribution}.

\subsubsection{Complementary Distribution}

Two words are \emph{in complementary distribution} if their distributions do not intersect. Complementary distribution
implies identical syntactic category.

\subsection{Classification of Syntactic Categories}

We classify the syntactic categories noted above in two ways.

\subsubsection{Open and Closed Classes}

Syntactic categories that permit new members are \emph{open classes}, and those that do not are \emph{closed classes}.
N, V, Adj, and Adv are open classes; and P, D, Conj, C, T, and Neg are closed classes.

\subsubsection{Lexical and Functional Categories}

Syntactic categories with full semantic content are \emph{lexical categories}, and those without are \emph{functional
categories}. Lexical categories coincide with open classes, and functional categories coincide with closed classes.

\subsection{Subcategories and Features}

D and T have \emph{subcategories}. The subcategories of D are
\begin{enumerate}
  \item \emph{articles},
  \item \emph{deictic articles},
  \item \emph{quantifiers},
  \item \emph{numerals},
  \item \emph{possessive pronouns}, and
  \item some \textit{wh-}question words;
\end{enumerate}
and the subcategories of T are
\begin{enumerate}
  \item \emph{auxiliaries},
  \item \emph{modals}, and
  \item non-finite tense marker.
\end{enumerate}

\Edc
