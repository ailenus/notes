\documentclass[12pt]{article}

\makeatletter
\def\@maketitle {
    \newpage
    \null
    \vskip 2em
    \begin{center}
        \let \footnote \thanks
        {
            \Large\bfseries \MakeUppercase{\@title} \par
        }
        \vskip 1.5em
        {
            \large \lineskip .5em \begin{tabular}[t]{c}
                \MakeUppercase{\@author}
            \end{tabular} \par
        }
        \vskip 1.5em
        {
            \large\scshape \@date
        }
    \end{center}
    \par
    \vskip 1.5em
}
\makeatother

%%%%%%%%%%%%%%%%%%%%%%%%%%%%%%%%%%%%%%%%%%%%%%%%%%%%%%%%%%%%%%%%%%%%%%%%%%%%%%%
%% Packages
%%%%%%%%%%%%%%%%%%%%%%%%%%%%%%%%%%%%%%%%%%%%%%%%%%%%%%%%%%%%%%%%%%%%%%%%%%%%%%%

%% basic
\usepackage[a4paper,margin=1in]{geometry}
\usepackage{fontspec}
\usepackage{polyglossia}

%% formatting
\usepackage{setspace}
\usepackage{enumitem}
\usepackage{tocloft}
\usepackage{titlesec}
\usepackage{fancyhdr}

%% maths
\usepackage{amsmath}
\usepackage{amsthm}
\usepackage{mathtools}
\usepackage[math-style=ISO,bold-style=ISO]{unicode-math}

%% utilities
\usepackage{array}
\usepackage{booktabs}
\usepackage[colorlinks=true]{hyperref}

%%%%%%%%%%%%%%%%%%%%%%%%%%%%%%%%%%%%%%%%%%%%%%%%%%%%%%%%%%%%%%%%%%%%%%%%%%%%%%%
%% Settings
%%%%%%%%%%%%%%%%%%%%%%%%%%%%%%%%%%%%%%%%%%%%%%%%%%%%%%%%%%%%%%%%%%%%%%%%%%%%%%%

%% basic
\setmainlanguage[variant=british]{english}
\SetLanguageKeys{english}{indentfirst=true}
\setmainfont{STIX Two Text}
\setmathfont{STIX Two Math}

%% formatting
\doublespacing
\setlist{noitemsep}
\setlist[1]{labelindent=\parindent}
\setlist[1]{listparindent=\parindent}
\setlist[enumerate,1]{label=(\alph*)}
\setlist[enumerate,2]{label=(\roman*)}
\setlength{\headheight}{15.2pt}
\setlength{\cftsecindent}{0em}
\setlength{\cftsubsecindent}{0em}
\setlength{\cftsubsubsecindent}{0em}
\renewcommand*{\cfttoctitlefont}{\large\scshape}
\renewcommand*{\cftsecfont}{\normalfont}
\renewcommand*{\cftsecpagefont}{\normalfont}
\renewcommand*{\cftsecleader}{\cftdotfill{\cftdotsep}}
\titleformat{\section}{
    \normalfont\Large\scshape\filcenter
}{}{1em}{}
\titleformat{\subsection}{
    \normalfont\large\scshape\raggedright
}{\thesubsection}{1em}{}
\titleformat{\subsubsection}{
    \normalfont\normalsize\scshape\raggedright
}{\thesubsubsection}{1em}{}
\setcounter{footnote}{1}

%% maths
\makeatletter
\newtheoremstyle{dfn}
    {\topsep}
    {\topsep}
    {\upshape\addtolength{\@totalleftmargin}{3em}
    \addtolength{\linewidth}{-6em}
    \parshape 1 3em \linewidth}
    {0pt}
    {\scshape}
    {.}
    {5pt plus 1pt minus 1pt}
    {}
\newtheoremstyle{thm}
    {\topsep}
    {\topsep}
    {\slshape\addtolength{\@totalleftmargin}{3em}
    \addtolength{\linewidth}{-6em}
    \parshape 1 3em \linewidth}
    {0pt}
    {\bfseries}
    {.}
    {5pt plus 1pt minus 1pt}
    {}
\newtheoremstyle{xrc}
    {\topsep}
    {\topsep}
    {\upshape\addtolength{\@totalleftmargin}{3em}
    \addtolength{\linewidth}{-6em}
    \parshape 1 3em \linewidth}
    {0pt}
    {\bfseries\scshape}
    {.}
    {5pt plus 1pt minus 1pt}
    {}
\newtheoremstyle{sol}
    {\topsep}
    {\topsep}
    {\upshape}
    {0pt}
    {\itshape}
    {.}
    {5pt plus 1pt minus 1pt}
    {}
\makeatother
\theoremstyle{dfn}
\newtheorem{dfn}{Definition}
\theoremstyle{thm}
\newtheorem{thm}{Theorem}
\newtheorem{lem}{Lemma}
\newtheorem{cor}{Corollary}
\renewcommand\qedsymbol{\(\mdlgwhtsquare\)}
\theoremstyle{xrc}
\newtheorem{xrc}{Exercise}
\theoremstyle{sol}
\newtheorem*{XxmpX}{Solution}
\newenvironment{solution}{
    \renewcommand{\qedsymbol}{\(\mdlgwhtlozenge\)}
    \pushQED{\qed}\begin{XxmpX}
}{\popQED\end{XxmpX}}
\def\equationautorefname~#1\null{(#1)\null}
\newcommand{\dfnautorefname}{definition}
\newcommand{\thmautorefname}{theorem}
\newcommand{\lemautorefname}{lemma}
\newcommand{\corautorefname}{corollary}

%% aliases
\newcommand*{\Bdc}[2][\today]{
    \title{#2}
    \author{Yannan Mao}
    \date{#1}
    \lhead{\scshape #2}
    \rhead{\scshape Yannan Mao}
    \begin{document}
    \maketitle
    \tableofcontents
    \thispagestyle{empty}
    \clearpage
    \pagestyle{fancy}
    \setcounter{page}{1}
}
\newcommand*{\Edc}{\end{document}}
\newcommand*{\Bdf}{\begin{dfn}}
\newcommand*{\Edf}{\end{dfn}}
\newcommand*{\Bth}{\begin{thm}}
\newcommand*{\Eth}{\end{thm}}
\newcommand*{\Blm}{\begin{lem}}
\newcommand*{\Elm}{\end{lem}}
\newcommand*{\Bcr}{\begin{cor}}
\newcommand*{\Ecr}{\end{cor}}
\newcommand*{\Bxr}{\begin{xrc}}
\newcommand*{\Exr}{\end{xrc}}
\newcommand*{\Bpr}{\begin{proof}}
\newcommand*{\Epr}{\end{proof}}
\newcommand*{\Bsl}{\begin{solution}}
\newcommand*{\Esl}{\end{solution}}

\DeclareMathOperator{\dom}{dom}             % domain
\DeclareMathOperator{\im}{im}               % image
\DeclareMathOperator*{\argmax}{arg\,max}    % arg max
\DeclareMathOperator*{\argmin}{arg\,min}    % arg min

\DeclarePairedDelimiter{\abs}{\lvert}{\rvert}               % absolute value, cardinality
\DeclarePairedDelimiter{\norm}{\lVert}{\rVert}              % norm
\DeclarePairedDelimiterX{\inn}[2]{\langle}{\rangle}{#1,#2}  % inner product
\DeclarePairedDelimiter{\set}{\{}{\}}                       % set

\newcommand*{\ee}{\ensuremath{\mathrm{e}}}              % e
\newcommand*{\ii}{\ensuremath{\mathrm{i}}}              % i
\newcommand*{\diff}{\ensuremath{\mathop{}\!\mathrm{d}}} % differential

\newcommand*{\pow}{\ensuremath{\mathcal{P}}}            % power set
\newcommand*{\nset}{\ensuremath{\emptyset}}             % null set
\newcommand*{\setnat}{\ensuremath{\mathbb{N}}}          % set of natural numbers
\newcommand*{\setint}{\ensuremath{\mathbb{Z}}}          % set of integers
\newcommand*{\setrat}{\ensuremath{\mathbb{Q}}}          % set of rational numbers
\newcommand*{\setreal}{\ensuremath{\mathbb{R}}}         % set of real numbers
\newcommand*{\setcomp}{\ensuremath{\mathbb{C}}}         % set of complex numbers
\newcommand*{\posint}{\ensuremath{\setint_{>0}}}        % set of positive integers
\newcommand*{\posreal}{\ensuremath{\setreal_{>0}}}      % set of positive real numbers
\newcommand*{\nonneg}{\ensuremath{\setreal_{\ge 0}}}    % set of nonnegative numbers
\newcommand*{\Nln}[1][n]{\ensuremath{\setnat_{<#1}}}    % set of natural numbers less than n

\newcommand*{\vct}[1]{\ensuremath{\symbf{#1}}}      % vector
\newcommand*{\map}[3]{\ensuremath{#1\colon#2\to#3}} % map
\newcommand*{\grad}{\ensuremath{\nabla}}            % gradient

\newcommand*{\dis}{\ensuremath{\displaystyle}}


\usepackage{lilyglyphs}
\usepackage{lyluatex}

\Bdc{Notes on Tonality}

\section{Fundamentals}

\subsection{Metres}

In music, a steady and regular pulse is a \emph{beat}, which may be \emph{strong} or \emph{weak} and are grouped into
\emph{bars} or \emph{measures} separated by \emph{bar lines}. The first beat of each bar is accented and referred to as
the \emph{downbeat}.

An arrangement of rhythm into a pattern of strong and weak beats is a \emph{metre}, which is defined by a \emph{time
signature}, consisting of two stacked numbers. The bottom number determines the note value of each beat, and the top
number determines the number of beats in each bar. The metre is \emph{simple} if the beat unit is not dotted and
\emph{compound} otherwise.

\subsubsection{Simple Metres}

In simple metres, the quarter note is the most common beat, and the eighth note and the half note may also function as
the beat. The quadruple simple metre with the quarter note as the beat \lilyTimeSignature{4}{4} is known as \emph{common
time} and written \lilyTimeC. The duple simple metre with the half note as the beat \lilyTimeSignature{2}{2} is known as
\emph{alla breve} or \emph{cut common time} and written \lilyTimeCHalf.

\Edc
