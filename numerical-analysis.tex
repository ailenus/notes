\documentclass[12pt]{article}

\makeatletter
\def\@maketitle {
    \newpage
    \null
    \vskip 2em
    \begin{center}
        \let \footnote \thanks
        {
            \Large\bfseries \MakeUppercase{\@title} \par
        }
        \vskip 1.5em
        {
            \large \lineskip .5em \begin{tabular}[t]{c}
                \MakeUppercase{\@author}
            \end{tabular} \par
        }
        \vskip 1.5em
        {
            \large\scshape \@date
        }
    \end{center}
    \par
    \vskip 1.5em
}
\makeatother

%%%%%%%%%%%%%%%%%%%%%%%%%%%%%%%%%%%%%%%%%%%%%%%%%%%%%%%%%%%%%%%%%%%%%%%%%%%%%%%
%% Packages
%%%%%%%%%%%%%%%%%%%%%%%%%%%%%%%%%%%%%%%%%%%%%%%%%%%%%%%%%%%%%%%%%%%%%%%%%%%%%%%

%% basic
\usepackage[a4paper,margin=1in]{geometry}
\usepackage{fontspec}
\usepackage{polyglossia}

%% formatting
\usepackage{setspace}
\usepackage{enumitem}
\usepackage{tocloft}
\usepackage{titlesec}
\usepackage{fancyhdr}

%% maths
\usepackage{amsmath}
\usepackage{amsthm}
\usepackage{mathtools}
\usepackage[math-style=ISO,bold-style=ISO]{unicode-math}

%% utilities
\usepackage{array}
\usepackage{booktabs}
\usepackage[colorlinks=true]{hyperref}

%%%%%%%%%%%%%%%%%%%%%%%%%%%%%%%%%%%%%%%%%%%%%%%%%%%%%%%%%%%%%%%%%%%%%%%%%%%%%%%
%% Settings
%%%%%%%%%%%%%%%%%%%%%%%%%%%%%%%%%%%%%%%%%%%%%%%%%%%%%%%%%%%%%%%%%%%%%%%%%%%%%%%

%% basic
\setmainlanguage[variant=british]{english}
\SetLanguageKeys{english}{indentfirst=true}
\setmainfont{STIX Two Text}
\setmathfont{STIX Two Math}

%% formatting
\doublespacing
\setlist{noitemsep}
\setlist[1]{labelindent=\parindent}
\setlist[1]{listparindent=\parindent}
\setlist[enumerate,1]{label=(\alph*)}
\setlist[enumerate,2]{label=(\roman*)}
\setlength{\headheight}{15.2pt}
\setlength{\cftsecindent}{0em}
\setlength{\cftsubsecindent}{0em}
\setlength{\cftsubsubsecindent}{0em}
\renewcommand*{\cfttoctitlefont}{\large\scshape}
\renewcommand*{\cftsecfont}{\normalfont}
\renewcommand*{\cftsecpagefont}{\normalfont}
\renewcommand*{\cftsecleader}{\cftdotfill{\cftdotsep}}
\titleformat{\section}{
    \normalfont\Large\scshape\filcenter
}{}{1em}{}
\titleformat{\subsection}{
    \normalfont\large\scshape\raggedright
}{\thesubsection}{1em}{}
\titleformat{\subsubsection}{
    \normalfont\normalsize\scshape\raggedright
}{\thesubsubsection}{1em}{}
\setcounter{footnote}{1}

%% maths
\makeatletter
\newtheoremstyle{dfn}
    {\topsep}
    {\topsep}
    {\upshape\addtolength{\@totalleftmargin}{3em}
    \addtolength{\linewidth}{-6em}
    \parshape 1 3em \linewidth}
    {0pt}
    {\scshape}
    {.}
    {5pt plus 1pt minus 1pt}
    {}
\newtheoremstyle{thm}
    {\topsep}
    {\topsep}
    {\slshape\addtolength{\@totalleftmargin}{3em}
    \addtolength{\linewidth}{-6em}
    \parshape 1 3em \linewidth}
    {0pt}
    {\bfseries}
    {.}
    {5pt plus 1pt minus 1pt}
    {}
\newtheoremstyle{xrc}
    {\topsep}
    {\topsep}
    {\upshape\addtolength{\@totalleftmargin}{3em}
    \addtolength{\linewidth}{-6em}
    \parshape 1 3em \linewidth}
    {0pt}
    {\bfseries\scshape}
    {.}
    {5pt plus 1pt minus 1pt}
    {}
\newtheoremstyle{sol}
    {\topsep}
    {\topsep}
    {\upshape}
    {0pt}
    {\itshape}
    {.}
    {5pt plus 1pt minus 1pt}
    {}
\makeatother
\theoremstyle{dfn}
\newtheorem{dfn}{Definition}
\theoremstyle{thm}
\newtheorem{thm}{Theorem}
\newtheorem{lem}{Lemma}
\newtheorem{cor}{Corollary}
\renewcommand\qedsymbol{\(\mdlgwhtsquare\)}
\theoremstyle{xrc}
\newtheorem{xrc}{Exercise}
\theoremstyle{sol}
\newtheorem*{XxmpX}{Solution}
\newenvironment{solution}{
    \renewcommand{\qedsymbol}{\(\mdlgwhtlozenge\)}
    \pushQED{\qed}\begin{XxmpX}
}{\popQED\end{XxmpX}}
\def\equationautorefname~#1\null{(#1)\null}
\newcommand{\dfnautorefname}{definition}
\newcommand{\thmautorefname}{theorem}
\newcommand{\lemautorefname}{lemma}
\newcommand{\corautorefname}{corollary}

%% aliases
\newcommand*{\Bdc}[2][\today]{
    \title{#2}
    \author{Yannan Mao}
    \date{#1}
    \lhead{\scshape #2}
    \rhead{\scshape Yannan Mao}
    \begin{document}
    \maketitle
    \tableofcontents
    \thispagestyle{empty}
    \clearpage
    \pagestyle{fancy}
    \setcounter{page}{1}
}
\newcommand*{\Edc}{\end{document}}
\newcommand*{\Bdf}{\begin{dfn}}
\newcommand*{\Edf}{\end{dfn}}
\newcommand*{\Bth}{\begin{thm}}
\newcommand*{\Eth}{\end{thm}}
\newcommand*{\Blm}{\begin{lem}}
\newcommand*{\Elm}{\end{lem}}
\newcommand*{\Bcr}{\begin{cor}}
\newcommand*{\Ecr}{\end{cor}}
\newcommand*{\Bxr}{\begin{xrc}}
\newcommand*{\Exr}{\end{xrc}}
\newcommand*{\Bpr}{\begin{proof}}
\newcommand*{\Epr}{\end{proof}}
\newcommand*{\Bsl}{\begin{solution}}
\newcommand*{\Esl}{\end{solution}}

\DeclareMathOperator{\dom}{dom}             % domain
\DeclareMathOperator{\im}{im}               % image
\DeclareMathOperator*{\argmax}{arg\,max}    % arg max
\DeclareMathOperator*{\argmin}{arg\,min}    % arg min

\DeclarePairedDelimiter{\abs}{\lvert}{\rvert}               % absolute value, cardinality
\DeclarePairedDelimiter{\norm}{\lVert}{\rVert}              % norm
\DeclarePairedDelimiterX{\inn}[2]{\langle}{\rangle}{#1,#2}  % inner product
\DeclarePairedDelimiter{\set}{\{}{\}}                       % set

\newcommand*{\ee}{\ensuremath{\mathrm{e}}}              % e
\newcommand*{\ii}{\ensuremath{\mathrm{i}}}              % i
\newcommand*{\diff}{\ensuremath{\mathop{}\!\mathrm{d}}} % differential

\newcommand*{\pow}{\ensuremath{\mathcal{P}}}            % power set
\newcommand*{\nset}{\ensuremath{\emptyset}}             % null set
\newcommand*{\setnat}{\ensuremath{\mathbb{N}}}          % set of natural numbers
\newcommand*{\setint}{\ensuremath{\mathbb{Z}}}          % set of integers
\newcommand*{\setrat}{\ensuremath{\mathbb{Q}}}          % set of rational numbers
\newcommand*{\setreal}{\ensuremath{\mathbb{R}}}         % set of real numbers
\newcommand*{\setcomp}{\ensuremath{\mathbb{C}}}         % set of complex numbers
\newcommand*{\posint}{\ensuremath{\setint_{>0}}}        % set of positive integers
\newcommand*{\posreal}{\ensuremath{\setreal_{>0}}}      % set of positive real numbers
\newcommand*{\nonneg}{\ensuremath{\setreal_{\ge 0}}}    % set of nonnegative numbers
\newcommand*{\Nln}[1][n]{\ensuremath{\setnat_{<#1}}}    % set of natural numbers less than n

\newcommand*{\vct}[1]{\ensuremath{\symbf{#1}}}      % vector
\newcommand*{\map}[3]{\ensuremath{#1\colon#2\to#3}} % map
\newcommand*{\grad}{\ensuremath{\nabla}}            % gradient

\newcommand*{\dis}{\ensuremath{\displaystyle}}


\Bdc{Notes on Numerical Analysis}

\section{Approximation Theory}

\subsection{Review of Relevant Topics}

An \emph{algebraic structure} consists of one or more sets with one or more binary operations thereon. We denote an
algebraic structure by an \(n\)-tuple whose first element is the \emph{carrier set of the algebraic structure} and is
followed by other sets followed by the binary operations.

When there is no confusion, we refer to the algebraic structure by the carrier set thereof.

\subsubsection{Groups, Rings, and Fields}

\Bdf
  A \emph{group} is an ordered pair \((G, *)\) wherein
  \begin{enumerate}
    \item \(G\) is a set; and
    \item \(*\) is a binary operation on \(G\) which satisfies
    \begin{enumerate}
      \item \(*\) is closed;
      \item \(*\) is associative;
      \item there exists an identity element for \(*\); and
      \item for each each element in \(G\) there exists an inverse element thereof.
    \end{enumerate}
  \end{enumerate}

  If \(*\) is commutative, the group is \emph{abelian}.
\Edf

\Bdf
  A \emph{ring} is an ordered triple \((K, *, \star)\) wherein
  \begin{enumerate}
    \item \((K, *)\) is an abelian group; and
    \item \(\star\) is a binary operation on \(K\) which satisfies
    \begin{enumerate}
      \item \(\star\) is closed;
      \item \(\star\) is commutative;
      \item \(\star\) is associative; and
      \item \(\star\) is distributive over \(*\).
    \end{enumerate}
  \end{enumerate}

  If there exists an identity element for \(\star\), the ring is \emph{unital}.

  If \(\star\) is commutative, the ring is \emph{commutative}.
\Edf

\Bdf
  A \emph{field} is a unital commutative ring \((F, *, \star)\) which satisfies
  \begin{enumerate}
    \item the identity for \(*\) is distinct from the identity for \(\star\);
    \item for each element in \(F\) distinct from the identity for \(*\) there exists an inverse element thereof for
    \(\star\).
  \end{enumerate}
\Edf

The set \(\setint\) with addition and multiplication is a unital commutative ring which is not a field, referred to as
the \emph{ring of integers}.

The set \(\setcomp\) with addition and multiplication is a field. A \emph{subfield of the field \(\setcomp\)} is a set
\(F \subseteq \setcomp\) which is itself a field with addition and multiplication.

The set \(\setrat\) is a subfield of \(\setcomp\). Any subfield of \(\setcomp\) contains \(\setrat\).

If \((K, *, \star)\) is a ring, we may apply \(*\) to the identity for \(\star\) with itself finitely many times and
obtain the identity for \(*\). If this happens in \(K\), the least positive integer \(n\) such that applying \(*\) to
the identity for \(\star\) \(n\) times results in the identity for \(*\) is the \emph{characteristic of the ring \(K\)}.
If this does not happen in \(K\), then \(K\) is a \emph{ring of characteristic zero}.

Any subfield of \(\setcomp\) is of characteristic zero.

\subsubsection{Vector Spaces}

Suppose \((F, +, \cdot)\) is a field. We write \(\lambda \mu\) as a shorthand for \(\lambda \cdot \mu\) for any
\(\lambda, \mu \in F\). We denote by \(0\) the identity for \(+\) in \(F\), and \(1\) the identity for \(\cdot\) in
\(F\). We refer to \(0\) as \emph{zero}.

\Bdf
  A \emph{vector space} is an ordered quadruple \((V, F, +, \cdot)\) wherein
  \begin{enumerate}
    \item \(V\) is a set, whose elements are referred to as \emph{vectors};
    \item \(F\) is a field, whose elements are referred to as \emph{scalars};
    \item \(+\) is a binary operation on \(V\), referred to as \emph{vector addition}, which satisfies
    \begin{enumerate}
      \item \(+\) is closed;
      \item \(+\) is commutative;
      \item \(+\) is associative;
      \item there exists an identity element for \(+\), referred to as the \emph{zero vector} and denoted by
      \(\vct{0}\); and
      \item for each \(\vct{a} \in V\) there exists an inverse element thereof, denoted by \(-\vct{a}\);
    \end{enumerate}
    and
    \item \(\map{\cdot}{F \times V}{V}\) is a binary operation, referred to as \emph{scalar multiplication}, which
    satisfies
    \begin{enumerate}
      \item \(1 \cdot \vct{a}\) for each \(\vct{a} \in V\);
      \item \((\lambda \mu) \cdot \vct{a} = \lambda \cdot (\mu \cdot \vct{a})\) for any \(\lambda, \mu \in F\) and
      \(\vct{a} \in V\);
      \item \(\lambda \cdot (\vct{a} + \vct{b}) = \lambda \cdot \vct{a} + \lambda \cdot \vct{b}\) for any
      \(\lambda \in F\) and \(\vct{a}, \vct{b} \in V\); and
      \item \((\lambda + \mu) \cdot \vct{a} = \lambda \cdot \vct{a} + \mu \cdot \vct{a}\) for any \(\lambda, \mu \in F\)
      and \(\vct{a} \in V\).
    \end{enumerate}
  \end{enumerate}

  For vectors \(\vct{a}\) and \(\vct{b}\), we refer to \(\vct{a} + \vct{b}\) as the \emph{sum of \(\vct{a}\) and
  \(\vct{b}\)}. For a scalar \(\lambda\) and a vector \(\vct{a}\), we refer to \(\lambda \cdot \vct{a}\) as the
  \emph{product of \(\lambda\) and \(\vct{a}\)}.
\Edf

We write \(\lambda \vct{a}\) as a shorthand for the product of \(\lambda\) and \(\vct{a}\).

The same set of vectors \(V\) may be the carrier set of a number of distinct vector spaces. When it is desirable to
specify the field, we say \(V\) is a \emph{vector space over the field \(F\)}.

A \emph{subspace of a vector space \(V\)} is a set \(W \subseteq V\) which is itself a vector space over the same field
with the operations of vector addition and scalar multiplication on \(V\).

\subsubsection{Bases and Dimensions}

Suppose \(F\) is a field.

\Bdf
  Let \(V\) be a vector space over \(F\). A set \(S \subseteq V\) is \emph{linearly dependent} if there exist \((n
  + 1)\) distinct vectors \(\vct{a}_0, \ldots, \vct{a}_n \in S\) and \((n + 1)\) scalars \(\lambda_0, \ldots, \lambda_n
  \in F\), not all of which are zero, such that
  \[
    \sum_{i = 0}^{n} \lambda_i \vct{a}_i = \vct{0}.
  \]

  A set which is not linearly dependent is \emph{linearly independent}.

  If the set \(S\) is finite, we also say that the vectors in \(S\) are linearly dependent or independent.
\Edf

A set which contains a linearly dependent set is linearly dependent.

A subset of a linearly independent set is linearly independent.

A set which contains \(\vct{0}\), the zero vector, is linearly dependent.

A set is linearly independent if and only if each finite subset thereof is linearly independent.

\Bdf
  Let \(V\) be a vector space, and let \(S \subseteq V\). The \emph{subspace spanned by the set \(S\)} is the
  intersection \(W\) of all subspaces of \(V\) which contain \(S\).

  If \(S\) is finite, we say the vectors in \(S\) span \(W\).
\Edf

\Bdf
  Let \(V\) be a vector space. A \emph{basis for \(V\)} is a linearly independent set of vectors in \(V\) which spans
  \(V\).

  A vector space is \emph{finite-dimensional} if it has a finite basis, and \emph{infinite-dimensional} otherwise.
\Edf

It follows from the axiom of choice that every vector space has a basis.

If \(V\) is a finite-dimensional vector space, any two bases for \(V\) are of the same cardinality. This allows us to
define the \emph{dimension of a finite vector space} as the cardinality of any basis therefor.

\subsubsection{The \(n\)-Tuple Space and the Standard Basis Thereof}

Suppose \(F\) is a field.

\Bxm
  We denote by \(F^n\) the set of all \(n\)-tuples \(\vct{a} = (a_0, \ldots, a_{n - 1})\) of scalars \(a_i \in F\),
  \(0 \leq i < n\).

  If \(\vct{a}, \vct{b} \in F\) with \(\vct{a} = (a_0, \ldots, a_{n - 1})\) and \(\vct{b} = (b_0, \ldots, b_{n - 1})\),
  we define the sum of \(\vct{a}\) and \(\vct{b}\) by
  \[
    \vct{a} + \vct{b} = (a_0 + b_0, \ldots, a_{n - 1} + b_{n - 1}).
  \]
  And we define the product of \(\lambda \in F\) and \(\vct{a}\) by
  \[
    \lambda \vct{a} = (\lambda a_0, \ldots, \lambda a_{n - 1}).
  \]

  Thus defined, \(F^n\) is a vector space, referred to as the \emph{\(n\)-tuple space}.

  As a special case for when \(n = 1\), the field \(F\) itself is a vector space.
\Exm

\Bxm
  Let \(E \subseteq F^n\) consist of the vectors \(\vct{e}_0, \vct{e}_1, \ldots, \vct{e}_{n - 1}\) defined by
  \begin{align*}
    \vct{e}_0 & = (1, 0, \ldots, 0),\\
    \vct{e}_1 & = (0, 1, \ldots, 0),\\
    & \vdots\\
    \vct{e}_{n - 1} & = (0, 0, \ldots, 1).
  \end{align*}

  Thus defined, the set \(E\) is the \emph{standard basis of the \(n\)-tuple space \(F^n\)}.
\Exm

\subsubsection{Algebras}

Suppose \(F\) is a field.

\Bdf
  An \emph{algebra over the field \(F\)} is an ordered quintuple \((\mathcal{A}, F, +, \cdot, \times)\) wherein
  \begin{enumerate}
    \item \((\mathcal{A}, F, +, \cdot)\) is a vector space; and
    \item \(\times\) is a binary operation on \(\mathcal{A}\), referred to as \emph{vector multiplication}, which
    satisfies
    \begin{enumerate}
      \item \(\times\) is closed;
      \item \(\times\) is distributive over \(+\); and
      \item \(\lambda (\vct{a} \times \vct{b}) = (\lambda \vct{a}) \times \vct{b} = \vct{a} \times (\lambda \vct{b})\)
      for any \(\lambda \in F\) and \(\vct{a}, \vct{b} \in \mathcal{A}\).
    \end{enumerate}
  \end{enumerate}

  If there exists an identity element for \(\times\), the algebra is \emph{unital}.

  If \(\times\) is commutative, the algebra is \emph{commutative}.

  If \(\times\) is associative, the algebra is \emph{associative}.

  For vectors \(\vct{a}\) and \(\vct{b}\), we refer to \(\vct{a} \times \vct{b}\) as the \emph{product of \(\vct{a}\)
  and \(\vct{b}\)}.
\Edf

\Bxm
  The set of \((n \times n)\) matrices over the field \(F\), with the usual matrix operations, is a unital associative
  algebra, which is not commutative if \(n > 1\).

  As a special case for when \(n = 1\), the field \(F\) itself is a unital commutative associative algebra.
\Exm

\subsubsection{Polynomials}

Suppose \(F\) is a field.

\Bxm
  Let \(S\) be a nonempty set, and let \(V\) be the set of functions from \(S\) into the field \(F\).

  We define the sum of \(f, g \in V\) by
  \[
    (f + g)(s) = f(s) + g(s)
  \]
  for each \(s \in S\). And we define the product of \(\lambda \in F\) and \(f \in V\) by
  \[
    (\lambda f)(s) = \lambda f(s)
  \]
  for each \(s \in S\).
  
  Thus defined, \(V\) is a vector space, referred to as the \emph{space of functions from the set \(S\) into the field
  \(F\)}.
\Exm

Note that \(F^n\), the \(n\)-tuple space, is a special case of the space of functions, if we take the set \(S\) to be
\(\Nln\), the set of natural numbers less than \(n\), and consider an \(n\)-tuple to be a function indexed by \(\Nln\).

We now consider the space of functions from \(\setnat\) into \(F\). We denote this space by \(F^\infty\). A vector in
\(F^\infty\) is thus an infinite sequence \(f = (f_0, f_1, f_2, \ldots)\) of scalars \(f_i \in F\), \(i \in \setnat\).

For two vectors \(f, g \in F^\infty\) with \(f = (f_0, f_1, f_2, \ldots)\) and \(g = (g_0, g_1, g_2, \ldots)\), we
define \(f g = f \times g\), the product of \(f\) and \(g\), by
\[
  (f g)_n = f_0 g_n + f_1 g_{n - 1} + f_2 g_{n - 2} + \cdots + f_n g_0 = \sum_{i = 0}^n f_i g_{n - i}.
\]
We may verify that \(F^\infty\) is an algebra with vector multiplication as defined above.

Note that, thus defined, vector multiplication is commutative and associative, and the vector \(1 = (1, 0, 0, \ldots)\)
is the vector-multiplicative identity. Therefore, \(F^\infty\) is a unital commutative associative algebra over the
field \(F\).

The vector \((0, 1, 0, \ldots)\) plays a distinguished role in \(F^\infty\), and we denote it by \(x\). The
product of \(x\) multiplied by itself \(n\) times is denoted by \(x^n\), and we put \(x^0 = 1\). Then, the vector
\(x^n\) is given by
\[
  x^n_i = \begin{cases}
    1, \text{ if } i = n,\\
    0, \text{ otherwise,}
  \end{cases}
\]
wherein \(x^n_i\) denotes the \(i\)-th entry of \(x^n\).

We observe that the set \(\set{x^n : n \in \setnat} \subseteq F^\infty\) is both independent and infinite. Thus, the
space \(F^\infty\) is infinite-dimensional.

The algebra \(F^\infty\) is referred to as the \emph{algebra of formal power series over \(F\)}. A vector \(f = (f_0,
f_1, f_2, \ldots) \in F^\infty\) is written
\[
  f = f_0 + f_1 x + f_2 x^2 + \cdots = \sum_{n = 0}^\infty f_n x^n.
\]

\Bdf
  Let \(F[x]\) be the subspace of \(F^\infty\) spanned by \(\set{x^n : n \in \setnat}\).

  Thus defined, a vector in \(F[x]\) is referred to as a \emph{polynomial over the field \(F\)}.
\Edf

A nonzero vector \(f \in F^\infty\) is a polynomial if and only if there exists an \(n \in \setnat\) such that \(f_n
\neq 0\) and that \(f_i = 0\) for each \(i > n\). This natural number \(n\) is unique, referred to as the \emph{degree
of the polynomial \(f\)} and denoted by \(\deg f\).

The degree of the zero vector \((0, 0, 0, \ldots)\), referred to as the \emph{zero polynomial} and denoted by \(0\), is
undefined.

If \(f \in F[x]\) is a nonzero polynomial of degree \(n\), it follows that
\[
  f = \sum_{i = 0}^n f_i x^i, \quad f_n \neq 0.
\]

Let \(f\) and \(g\) be nonzero polynomials over \(F\). Then \(f g\) is also a nonzero polynomial and \(\deg (f g)
= \deg f + \deg g\). If in addition \(f + g \neq 0\), then \(\deg (f + g) \leq \max\set{\deg f, \deg g}\).

\subsubsection{Roots of Polynomials}

Suppose \(F\) is a field.

\Blm
  \label{lm00}
  Let \(f\) and \(d\) be nonzero polynomials over the field \(F\) such that \(\deg f \geq \deg d\). Then there exists a
  polynomial \(g \in F[x]\) such that either \(f - d g = 0\) or \(\deg (f - d g) < \deg f\).
\Elm
\Bpr
  Suppose that
  \[
    f = f_m x^m + \sum_{i = 0}^{m - 1} f_i x^i, \quad f_m \neq 0,
  \]
  and that
  \[
    d = d_n x^n + \sum_{j = 0}^{n - 1} d_j x^j, \quad d_n \neq 0.
  \]
  Then \(m \geq n\), and either
  \[
    f - \frac{f_m}{d_n} x^{m - n} = 0
  \]
  or
  \[
    \deg \left(f - \frac{f_m}{d_n} x^{m - n} d\right) < \deg f.
  \]
  Hence, we may take \(\dis g = \frac{f_m}{d_n} x^{m - n}\), which satisfies the condition.
\Epr

\Bth
  \label{th00}
  If \(f\) and \(d\) are polynomials over the field \(F\) and \(d \neq 0\), then there exist a unique pair of
  polynomials \(q, r \in F[x]\) such that
  \begin{enumerate}
    \item \(f = d q + r\); and
    \item either \(r = 0\) or \(\deg r < \deg d\).
  \end{enumerate}
\Eth
\Bpr
  If \(f = 0\) or \(\deg f < \deg d\), then \(q = 0\) and \(r = f\).

  If \(f \neq 0\) and \(\deg f \geq \deg d\), by \autoref{lm00} we may choose a polynomial \(g_0\) such that \(f - d g_0
  = 0\) or \(\deg (f - d g_0) < \deg f\). If \(f - d g_0 = 0\), then \(q = g_0\) and \(r = 0\). If \(f - d g_0 \neq 0\)
  and \(\deg (f - d g_0) < \deg d\), then \(q = g_0\) and \(r = f - d g_0\).

  If \(f - d g_0 \neq 0\) and \(\deg (f - d g_0) \geq \deg d\), we apply \autoref{lm00} again and choose another
  polynomial \(g_1\) such that \(f - d (g_0 + g_1) = 0\) or \(\deg \big(f - d (g_0 + g_1)\big) < \deg (f - d g_0)\).
  Continuing this process, in the end we obtain polynomials \(q\) and \(r\) such that \(r = 0\) or \(\deg r < \deg d\)
  and that \(f = d q + r\).

  To prove that \(q\) and \(r\) are unique, suppose for the sake of contradiction we also have \(f = d q' + r'\) with
  \(r' = 0\) or \(\deg r' < \deg d\). It follows that \(d q + r = d q' + r'\), and then \(d (q - q') = r' - r\). Because
  \(q - q' \neq 0\), we have \(d (q - q') \neq 0\), and so \(\deg d + \deg (q - q') = \deg (r' - r)\). This is a
  contradiction, as \(\deg (r' - r) < \deg d\). Ergo, our supposition is false.

  Hence, the theorem holds.
\Epr

\Bdf
  Let \(d\) be a nonzero polynomial over the field \(F\). If \(f \in F[x]\), then by \autoref{th00} there exists at most
  one polynomial \(q \in F[x]\) such that \(f = d q\). If such a \(q\) exists, we say that \emph{\(d\) divides \(f\)},
  that \emph{\(f\) is divisible by \(d\)}, that \(d\) is a \emph{divisor of \(f\)}, that \(f\) is a \emph{multiple of
  \(d\)}, and that \(q\) is the \emph{quotient of \(f\) and \(d\)}; and we write \(d \, | \, f\) and \(q = f / d\).
\Edf

\Bdf
  Let \(f\) be a polynomial over the field \(F\). A scalar \(c \in F\) is a \emph{root of \(f\) over the field \(F\)} if
  \(f(c) = 0\).
\Edf

\Bcr
  Let \(f\) be a polynomial over the field \(F\), and let \(c \in F\). Then \((x - c) \, | \, f\) if and only if \(c\)
  is a root of \(f\).
\Ecr
\Bpr
  By \autoref{th00}, \(f = (x - c) q + r\) wherein \(r\) is a scalar polynomial. Then, \(f(c) = r(c)\). It follows that
  \(r = 0\) if and only if \(f(c) = 0\).
\Epr

If \(c\) is a root of a polynomial \(f\), the \emph{multiplicity of \(c\) as a root of \(f\)} is the largest positive
integer \(n\) such that \((x - c)^n \, | \, f\).

The multiplicity of any root of \(f\) is at most \(\deg f\).

Thus concludes our review of relevant topics.

\subsection{Bézier Curves}

We investigate polynomials in the \(d\)-dimensional Euclidean space \(\setreal^d\).

\Bdf
  A \emph{polynomial of degree \(n\) in \(\setreal^d\)} is a function \(\map{\vct{p}}{\setreal}{\setreal^d}\) defined by
  \[
    \vct{p}(t) = \sum_{i = 0}^n \vct{a}_i t_i, \quad \vct{a}_i \in \setreal^d \text{ and } \vct{a}_n \neq \vct{0}.
  \]

  We denote the space of polynomials of degree less than or equal to \(n\) in \(\setreal^d\) by \(P_n^d\).

  As a special case for when \(d = 1\), the space is denoted by \(P_n\) and a polynomial \(p \in P_n\) is given by
  \[
    p(t) = \sum_{i = 0}^n a_i t_i, \quad a_i \in \setreal.
  \]
\Edf

Let \(\set{\vct{e}_0, \ldots, \vct{e}_{d - 1}}\) be the standard basis of \(\setreal^d\). If \(\set{p_0, \ldots, p_n}\)
is a basis for \(P_n\), then the polynomials
\[
  \set{p_i \vct{e}_j : i \in \setnat_{\leq n} \text{ and } j \in \setnat_{< d}}
\]
form a basis for \(P_n^d\).

% The graph \(\Gamma_{\vct{p}}\) of a polynomial \(\vct{p} \in P_n^d\), given by
% \[
%   \map{\Gamma_{\vct{p}}}{\setreal}{\setreal^{d + 1}}, \quad t \mapsto \big(t, \vct{p}(t)\big),
% \]
% may now be considered a polynomial \(\Gamma_{\vct{p}} \in P_n^{d + 1}\).

% If \(\vct{p} \in P_n^d\) is given in coefficient representation
% \[
%   \vct{p}(t) = \vct{a}_0 + \vct{a}_1 t + \vct{a}_2 t^2 + \cdots + \vct{a}_n t^n = \sum_{i = 0}^n \vct{a}_i t_i,
% \]
% then the graph of \(\vct{p}\) is
% \[
%   \Gamma_{\vct{p}}(t) = \binom{0}{\vct{a}_0} + \binom{1}{\vct{a}_1} t + \binom{0}{\vct{a}_2} t^2 + \cdots
%   + \binom{0}{\vct{a}_n} t^n
% \]

\subsubsection{Berstein Basis Polynomials}

We denote the closed interval between any \(a, b \in \setreal\) by \([a, b]\). Namely,
\[
  [a, b] = \set{x : x = \lambda a + (1 - \lambda) b \text{ and } 0 \leq \lambda \leq 1}.
\]

\Bdf
  The \emph{\(i\)-th Bernstein basis polynomial of degree \(n\)} is the polynomial \(B_{i, n} \in P_n\) defined by
  \[
    B_{i, n}(t) = \binom{n}{i} (1 - t)^{n - i} t^i
  \]
  for each \(t \in [0, 1]\), wherein \(i \in \setnat_{\leq n}\).
\Edf

\Bxm
  The first few Bernstein polynomials are
  \begin{align*}
    B_{0, 0}(t) & = 1,\\
    B_{0, 1}(t) & = 1 - t, & B_{1, 1}(t) & = t,\\
    B_{0, 2}(t) & = (1 - t)^2, & B_{1, 2}(t) & = 2 t (1 - t), & B_{2, 2}(t) & = t^2,\\
    B_{0, 3}(t) & = (1 - t)^3, & B_{1, 3}(t) & = 3 t (1 - t)^2, & B_{2, 3}(t) & = 3 t^2 (1 - t), & B_{3, 3}(t) & = t^3.
  \end{align*}
\Exm

For any closed interval \([a, b]\), we can apply to it an affine transformation onto the unit interval \([0, 1]\) by
\[
  t \mapsto \frac{t - a}{b - a}.
\]
Then, we define Bernstein basis polynomials with respect to any closed interval \([a, b]\) by
\[
  B_{i, n}(t; a, b) = B_{i, n}\left(\frac{t - a}{b - a}\right) = \frac{1}{(b - a)^n} \binom{n}{i} (t - a)^i (b - t)^{n
  - i}
\]
for each \(t \in [a, b]\), wherein \(i \in \setnat_{\leq n}\).

\Bth
  The Bernstein basis polynomials \(B_{i, n}\) satisfy the following properties:
  \begin{enumerate}
    \item \(0\) is a root of multiplicity \(i\) of \(B_{i, n}\).
    \item \(1\) is a root of multiplicity \((n - i)\) of \(B_{i, n}\).
    \item \(B_{i, n}(t) = B_{n - i, n}(1 - t)\) for each \(t \in [0, 1]\).
    \item \((1 - t) B_{0, n} = B_{n, n + 1}\) for each \(t \in [0, 1]\).
    \item \(t B_{n, n} = B_{n + 1, n + 1}\) for each \(t \in [0, 1]\).
    \item The \((n + 1)\) Bernstein basis polynomials \(B_{i, n}\), \(i \in \setnat_{\leq n}\), are nonnegative on
    \([0, 1]\) and form a partition of unity; i.e., \(B_{i, n}(t) \geq 0\) and
    \[
      \sum_{i = 0}^n B_{i, n}(t) = 1
    \]
    for each \(t \in [0, 1]\).
    \item \(B_{i, n}\) has a unique local maximum at \(\frac {i}{n}\).
    \item The recurrence relation
    \[
      B_{i + 1, n}(t) = t B_{i, n - 1}(t) + (1 - t) B_{i + 1, n - 1}(t)
    \]
    for each \(t \in [0, 1]\) and \(i \in \Nln\) holds.
    \item The \((n + 1)\) Bernstein basis polynomials \(B_{i, n}\), \(i \in \setnat_{\leq n}\), form a basis for
    \(P_n\).
  \end{enumerate}
\Eth
\Bpr
  TODO % TODO
\Epr

Because the \((n + 1)\) Bernstein basis polynomials \(B_{i, n}\), \(i \in \setnat_{\leq n}\), form a basis for \(P_n\),
the polynomials
\[
  \set{B_{i, n} \vct{e}_j : i \in \setnat_{\leq n} \text{ and } j \in \setnat_{< d}}
\]
form a basis for \(P_n^d\). Thus, a polynomial \(\vct{p} \in P_n^d\) may be written
\[
  \vct{p} = \sum_{i = 0}^n B_{i, n} \vct{\beta}_i, \quad \vct{\beta}_i \in \setreal^d,
\]
wherein each coefficient \(\vct{\beta}_i\), \(i \in \setnat_{\leq n}\), is given by
\[
  \vct{\beta}_i = \sum_{j = 0}^{d - 1} \lambda_j \vct{e}_j
\]
for some \(\lambda_j \in \setreal\), \(j \in \setnat_{< d}\). The coefficients \(\vct{\beta}_j\) are referred to as
\emph{Bézier coefficients} and the \emph{control points of \(\vct{p}\)}.

\Edc
